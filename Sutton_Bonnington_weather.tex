% Options for packages loaded elsewhere
\PassOptionsToPackage{unicode}{hyperref}
\PassOptionsToPackage{hyphens}{url}
%
\documentclass[
  ignorenonframetext,
]{beamer}
\title{VC55 Weather Case Study}
\author{Paul J. Palmer}
\date{}

\usepackage{pgfpages}
\setbeamertemplate{caption}[numbered]
\setbeamertemplate{caption label separator}{: }
\setbeamercolor{caption name}{fg=normal text.fg}
\beamertemplatenavigationsymbolsempty
% Prevent slide breaks in the middle of a paragraph
\widowpenalties 1 10000
\raggedbottom
\setbeamertemplate{part page}{
  \centering
  \begin{beamercolorbox}[sep=16pt,center]{part title}
    \usebeamerfont{part title}\insertpart\par
  \end{beamercolorbox}
}
\setbeamertemplate{section page}{
  \centering
  \begin{beamercolorbox}[sep=12pt,center]{part title}
    \usebeamerfont{section title}\insertsection\par
  \end{beamercolorbox}
}
\setbeamertemplate{subsection page}{
  \centering
  \begin{beamercolorbox}[sep=8pt,center]{part title}
    \usebeamerfont{subsection title}\insertsubsection\par
  \end{beamercolorbox}
}
\AtBeginPart{
  \frame{\partpage}
}
\AtBeginSection{
  \ifbibliography
  \else
    \frame{\sectionpage}
  \fi
}
\AtBeginSubsection{
  \frame{\subsectionpage}
}
\usepackage{amsmath,amssymb}
\usepackage{lmodern}
\usepackage{iftex}
\ifPDFTeX
  \usepackage[T1]{fontenc}
  \usepackage[utf8]{inputenc}
  \usepackage{textcomp} % provide euro and other symbols
\else % if luatex or xetex
  \usepackage{unicode-math}
  \defaultfontfeatures{Scale=MatchLowercase}
  \defaultfontfeatures[\rmfamily]{Ligatures=TeX,Scale=1}
\fi
\usetheme[]{AnnArbor}
\usecolortheme{dolphin}
\usefonttheme{structurebold}
% Use upquote if available, for straight quotes in verbatim environments
\IfFileExists{upquote.sty}{\usepackage{upquote}}{}
\IfFileExists{microtype.sty}{% use microtype if available
  \usepackage[]{microtype}
  \UseMicrotypeSet[protrusion]{basicmath} % disable protrusion for tt fonts
}{}
\makeatletter
\@ifundefined{KOMAClassName}{% if non-KOMA class
  \IfFileExists{parskip.sty}{%
    \usepackage{parskip}
  }{% else
    \setlength{\parindent}{0pt}
    \setlength{\parskip}{6pt plus 2pt minus 1pt}}
}{% if KOMA class
  \KOMAoptions{parskip=half}}
\makeatother
\usepackage{xcolor}
\IfFileExists{xurl.sty}{\usepackage{xurl}}{} % add URL line breaks if available
\IfFileExists{bookmark.sty}{\usepackage{bookmark}}{\usepackage{hyperref}}
\hypersetup{
  pdftitle={VC55 Weather Case Study},
  pdfauthor={Paul J. Palmer},
  hidelinks,
  pdfcreator={LaTeX via pandoc}}
\urlstyle{same} % disable monospaced font for URLs
\newif\ifbibliography
\usepackage{color}
\usepackage{fancyvrb}
\newcommand{\VerbBar}{|}
\newcommand{\VERB}{\Verb[commandchars=\\\{\}]}
\DefineVerbatimEnvironment{Highlighting}{Verbatim}{commandchars=\\\{\}}
% Add ',fontsize=\small' for more characters per line
\usepackage{framed}
\definecolor{shadecolor}{RGB}{248,248,248}
\newenvironment{Shaded}{\begin{snugshade}}{\end{snugshade}}
\newcommand{\AlertTok}[1]{\textcolor[rgb]{0.94,0.16,0.16}{#1}}
\newcommand{\AnnotationTok}[1]{\textcolor[rgb]{0.56,0.35,0.01}{\textbf{\textit{#1}}}}
\newcommand{\AttributeTok}[1]{\textcolor[rgb]{0.77,0.63,0.00}{#1}}
\newcommand{\BaseNTok}[1]{\textcolor[rgb]{0.00,0.00,0.81}{#1}}
\newcommand{\BuiltInTok}[1]{#1}
\newcommand{\CharTok}[1]{\textcolor[rgb]{0.31,0.60,0.02}{#1}}
\newcommand{\CommentTok}[1]{\textcolor[rgb]{0.56,0.35,0.01}{\textit{#1}}}
\newcommand{\CommentVarTok}[1]{\textcolor[rgb]{0.56,0.35,0.01}{\textbf{\textit{#1}}}}
\newcommand{\ConstantTok}[1]{\textcolor[rgb]{0.00,0.00,0.00}{#1}}
\newcommand{\ControlFlowTok}[1]{\textcolor[rgb]{0.13,0.29,0.53}{\textbf{#1}}}
\newcommand{\DataTypeTok}[1]{\textcolor[rgb]{0.13,0.29,0.53}{#1}}
\newcommand{\DecValTok}[1]{\textcolor[rgb]{0.00,0.00,0.81}{#1}}
\newcommand{\DocumentationTok}[1]{\textcolor[rgb]{0.56,0.35,0.01}{\textbf{\textit{#1}}}}
\newcommand{\ErrorTok}[1]{\textcolor[rgb]{0.64,0.00,0.00}{\textbf{#1}}}
\newcommand{\ExtensionTok}[1]{#1}
\newcommand{\FloatTok}[1]{\textcolor[rgb]{0.00,0.00,0.81}{#1}}
\newcommand{\FunctionTok}[1]{\textcolor[rgb]{0.00,0.00,0.00}{#1}}
\newcommand{\ImportTok}[1]{#1}
\newcommand{\InformationTok}[1]{\textcolor[rgb]{0.56,0.35,0.01}{\textbf{\textit{#1}}}}
\newcommand{\KeywordTok}[1]{\textcolor[rgb]{0.13,0.29,0.53}{\textbf{#1}}}
\newcommand{\NormalTok}[1]{#1}
\newcommand{\OperatorTok}[1]{\textcolor[rgb]{0.81,0.36,0.00}{\textbf{#1}}}
\newcommand{\OtherTok}[1]{\textcolor[rgb]{0.56,0.35,0.01}{#1}}
\newcommand{\PreprocessorTok}[1]{\textcolor[rgb]{0.56,0.35,0.01}{\textit{#1}}}
\newcommand{\RegionMarkerTok}[1]{#1}
\newcommand{\SpecialCharTok}[1]{\textcolor[rgb]{0.00,0.00,0.00}{#1}}
\newcommand{\SpecialStringTok}[1]{\textcolor[rgb]{0.31,0.60,0.02}{#1}}
\newcommand{\StringTok}[1]{\textcolor[rgb]{0.31,0.60,0.02}{#1}}
\newcommand{\VariableTok}[1]{\textcolor[rgb]{0.00,0.00,0.00}{#1}}
\newcommand{\VerbatimStringTok}[1]{\textcolor[rgb]{0.31,0.60,0.02}{#1}}
\newcommand{\WarningTok}[1]{\textcolor[rgb]{0.56,0.35,0.01}{\textbf{\textit{#1}}}}
\usepackage{longtable,booktabs,array}
\usepackage{calc} % for calculating minipage widths
\usepackage{caption}
% Make caption package work with longtable
\makeatletter
\def\fnum@table{\tablename~\thetable}
\makeatother
\setlength{\emergencystretch}{3em} % prevent overfull lines
\providecommand{\tightlist}{%
  \setlength{\itemsep}{0pt}\setlength{\parskip}{0pt}}
\setcounter{secnumdepth}{-\maxdimen} % remove section numbering
\title[ VC55 Weather Case Study]{VC55 Long Term Weather Data From Sutton Bonnington} % The short title appears at the bottom of every slide, the full title is only on the title page
\subtitle{A case study using data state to analyse inconvenient data}

%\author{Dr. Paul J. Palmer}

\institute[LU] % Your institution as it will appear on the bottom of every slide, may be shorthand to save space
{
    Visiting Fellow \\
    Loughborough University \\
    \copyright~P.J.~Palmer~2022
    
    
    
    % Your institution for the title page
}
\date{\today} % Date, can be changed to a custom date

  %title: ""
 % author: Dr. Paul J. Palmer
  %date:  "`r format(Sys.time(), '%d %B, %Y')`"
\ifLuaTeX
  \usepackage{selnolig}  % disable illegal ligatures
\fi

\begin{document}
\frame{\titlepage}

\begin{frame}{Introduction}
\protect\hypertarget{introduction}{}
\begin{itemize}
\tightlist
\item
  The purpose of this vignette is a practical demonstration of reusable
  templates based upon the novel concept of data state.
\item
  This report is used as a case study for the paper: \emph{Achieving
  Analytical Fluency With Complex Data} and uses real world long term
  weather data as its source.
\item
  It is not the intention to analyse climate change, but the trends
  uncovered are striking, even in this single public domain source.
\end{itemize}
\end{frame}

\begin{frame}[fragile]{Load Libraries}
\protect\hypertarget{load-libraries}{}
Load the Tidyverse libraries and other helpers before the analysis
starts.

\begin{Shaded}
\begin{Highlighting}[]
\FunctionTok{library}\NormalTok{(tidyverse) }\CommentTok{\# Load all the Tidyverse packages in one go.}
\FunctionTok{library}\NormalTok{(kableExtra) }\CommentTok{\# Enable advanced table styling}
\FunctionTok{library}\NormalTok{(qqplotr) }\CommentTok{\# Enable QQ plots for statistical analysis}
\FunctionTok{library}\NormalTok{(readr) }\CommentTok{\# Improved reading of text files}
\FunctionTok{library}\NormalTok{(timeDate) }\CommentTok{\# Helper for time date manipulation}
\FunctionTok{library}\NormalTok{(rprojroot) }\CommentTok{\# Useful utilities}
\FunctionTok{library}\NormalTok{(fs) }\CommentTok{\# System independent file paths}
\end{Highlighting}
\end{Shaded}
\end{frame}

\begin{frame}[fragile]{Read The Sutton Bonnington Data}
\protect\hypertarget{read-the-sutton-bonnington-data}{}
The absolute path helps to write code that is computer independent but
it does use the Rstudio specific function to find the current project
location. The use of \texttt{fs} to build the file path ensures
compatibility across all operating systems. It also gives a chance to
check that it is correct before loading the file. Note that the use of
\texttt{rprojroot::} and \texttt{fs::} is not strictly necessary as the
library has been loaded. This form explicitly specifies the installed
package from which the functions are called and is used here for
clarity.
\end{frame}

\begin{frame}[fragile]{Define The Data Path}
\protect\hypertarget{define-the-data-path}{}
\begin{Shaded}
\begin{Highlighting}[]
\NormalTok{absolute.path }\OtherTok{\textless{}{-}}\NormalTok{ rprojroot}\SpecialCharTok{::}\FunctionTok{find\_rstudio\_root\_file}\NormalTok{()}
\NormalTok{path.to.my.data }\OtherTok{\textless{}{-}}\NormalTok{ fs}\SpecialCharTok{::}\FunctionTok{path}\NormalTok{( absolute.path,}
                             \StringTok{"data{-}ext"}\NormalTok{,}
                             \StringTok{"suttonboningtondata"}\NormalTok{, }
                            \AttributeTok{ext =} \StringTok{"txt"}\NormalTok{)}
\end{Highlighting}
\end{Shaded}
\end{frame}

\begin{frame}[fragile]{Read the Data}
\protect\hypertarget{read-the-data}{}
The \texttt{read\_table} function does a good job of reading the data
into columns recognising the partitioning of fields with the use of
spaces after the first five lines of text are skipped. Although there
are numerous \textbf{warnings} comparing those rows with the final
output (allowing for the 5 row offset), uncovers no \textbf{errors}.
There are 7 data columns used and the warnings are generated by
occasional characters in column 8 which we ignore. The investigation of
such anomalies is an important part of the data reading process, and
while we do not need to know why they occur, their interpretation
affects the onward analysis.

\begin{Shaded}
\begin{Highlighting}[]
\NormalTok{suttonboningtondata }\OtherTok{\textless{}{-}} \FunctionTok{read\_table}\NormalTok{(}
\NormalTok{  path.to.my.data,}
  \AttributeTok{skip =} \DecValTok{5}\NormalTok{) }\CommentTok{\# skip 5 lines of text}
\end{Highlighting}
\end{Shaded}
\end{frame}

\begin{frame}[fragile]{Rework The Data}
\protect\hypertarget{rework-the-data}{}
Renaming the column names requires checking the results manually with
\texttt{view(suttonboningtondata)} before making a list of names.

\begin{Shaded}
\begin{Highlighting}[]
\CommentTok{\# Rename the columns to something useful}
\NormalTok{new.colnames }\OtherTok{\textless{}{-}} \FunctionTok{c}\NormalTok{( }\StringTok{"YYYY"}\NormalTok{,}
                   \StringTok{"mm"}\NormalTok{,}
                   \StringTok{"tmax.degC"}\NormalTok{,}
                   \StringTok{"tmin.degC"}\NormalTok{,}
                   \StringTok{"airfrost.days"}\NormalTok{,}
                   \StringTok{"rain.mm"}\NormalTok{,}
                   \StringTok{"sun.hours"}
\NormalTok{  )}
\FunctionTok{colnames}\NormalTok{(suttonboningtondata) }\OtherTok{\textless{}{-}}\NormalTok{ new.colnames}
\end{Highlighting}
\end{Shaded}

The use of spaces has been avoided since they can cause problems in
programmatic analysis.
\end{frame}

\begin{frame}[fragile]{Delete Spurious Rows}
\protect\hypertarget{delete-spurious-rows}{}
The text file headings were split across two rows so the first row of
the data contains fragments which we can drop since we have included
them in our column names manually.

\begin{Shaded}
\begin{Highlighting}[]
\CommentTok{\# Drop row 1}
\NormalTok{suttonboningtondata }\OtherTok{\textless{}{-}}\NormalTok{ suttonboningtondata[}\SpecialCharTok{{-}}\DecValTok{1}\NormalTok{,]}
\end{Highlighting}
\end{Shaded}
\end{frame}

\begin{frame}[fragile]{Remove Unwanted Characters}
\protect\hypertarget{remove-unwanted-characters}{}
We can now remove the unwanted characters that remain in the data. Note
the use of the double escape \texttt{\textbackslash{}\textbackslash{}*}
to select the asterisk. The removal of unwanted special characters is
often a problem in nascent data so it is best to get rid of them before
they cause problems when code is executed. Not how \texttt{tidyverse}`
syntax allows us to daisy chain the substitutions,

\begin{Shaded}
\begin{Highlighting}[]
\NormalTok{suttonboningtondata }\OtherTok{\textless{}{-}}\NormalTok{ suttonboningtondata }\SpecialCharTok{\%\textgreater{}\%}
                        \FunctionTok{map\_df}\NormalTok{( gsub,}
                                \AttributeTok{pattern =} \StringTok{"}\SpecialCharTok{\textbackslash{}\textbackslash{}}\StringTok{*"}\NormalTok{,}
                                \AttributeTok{replacement =} \StringTok{""}\NormalTok{) }\SpecialCharTok{\%\textgreater{}\%}
                        \FunctionTok{map\_df}\NormalTok{( gsub,}
                                \AttributeTok{pattern =} \StringTok{"{-}{-}{-}"}\NormalTok{,}
                                \AttributeTok{replacement =} \ConstantTok{NA}
\NormalTok{                                )}
\end{Highlighting}
\end{Shaded}
\end{frame}

\begin{frame}[fragile]{Problems With Dates}
\protect\hypertarget{problems-with-dates}{}
The early months of the year are represented by a single digit so we
need to pad them with a leading zero. The we can create a date in the
ISO recognised format: \texttt{YYYY-MM-DD}. As these are monthly
summaries we choose the last day of the month which allows us to convert
from a character string to a date when we plot results. We do this by
building a dummy. but correctly formatted date, and then finding the
last day of the month using the helper function from \texttt{timeDate}.
There are numerous in which this could have been calculated, but the R
language users have written many helpers for tedious tasks such as this.
\end{frame}

\begin{frame}[fragile]{Building The Nominal Dates}
\protect\hypertarget{building-the-nominal-dates}{}
\begin{Shaded}
\begin{Highlighting}[]
\NormalTok{suttonboningtondata}\SpecialCharTok{$}\NormalTok{mm }\OtherTok{\textless{}{-}}\NormalTok{   stringr}\SpecialCharTok{::}\FunctionTok{str\_pad}\NormalTok{(}
\NormalTok{  suttonboningtondata}\SpecialCharTok{$}\NormalTok{mm, }\AttributeTok{width =} \DecValTok{2}\NormalTok{, }\AttributeTok{pad =} \StringTok{"0"}\NormalTok{)}
 \CommentTok{\# Create a date at the end of the month.}
\NormalTok{dummyDate }\OtherTok{\textless{}{-}} \FunctionTok{with}\NormalTok{( suttonboningtondata, }
                  \FunctionTok{paste}\NormalTok{( YYYY, }
\NormalTok{                  mm, }
                  \StringTok{"28"}\NormalTok{, }\CommentTok{\# Need a dummy day}
                  \AttributeTok{sep=} \StringTok{"{-}"}\NormalTok{   ) }
\NormalTok{                  )}
\CommentTok{\# Calculates the last day of the month}
\NormalTok{ suttonboningtondata}\SpecialCharTok{$}\NormalTok{YYYYMMDD }\OtherTok{\textless{}{-}} 
\NormalTok{   timeDate}\SpecialCharTok{::}\FunctionTok{timeLastDayInMonth}\NormalTok{(dummyDate) }\SpecialCharTok{\%\textgreater{}\%} 
   \FunctionTok{as.character}\NormalTok{()}
\end{Highlighting}
\end{Shaded}
\end{frame}

\begin{frame}[fragile]{Introducing Datum Triples}
\protect\hypertarget{introducing-datum-triples}{}
The long datum triple format requires the explicit use of a unique
identifier for each row which we call \texttt{datumEntity}. When data is
in a wide format, as at present, the identifier is implicit as in the
row number, but this is a relative term that is affected by order. The
analysis that follows uses the advantages that are brought with an
explicit definition.

\begin{Shaded}
\begin{Highlighting}[]
\NormalTok{suttonboningtondata}\SpecialCharTok{$}\NormalTok{datumEntity }\OtherTok{\textless{}{-}} \FunctionTok{with}\NormalTok{(}
\NormalTok{  suttonboningtondata,}
  \FunctionTok{paste}\NormalTok{(}\StringTok{"Sutton.Bonnington"}\NormalTok{, YYYY, mm, }\AttributeTok{sep =} \StringTok{":"}\NormalTok{ )}
\NormalTok{  )}
\end{Highlighting}
\end{Shaded}
\end{frame}

\begin{frame}[fragile]{Adding Nominal Attributes}
\protect\hypertarget{adding-nominal-attributes}{}
There are also some data attributes which were described in the five
lines of text which apply to all rows of data. We manually add them here
as they will be useful if we combine our data with similar sources from
other locations.

\begin{Shaded}
\begin{Highlighting}[]
\CommentTok{\# manually name some values}
\NormalTok{suttonboningtondata}\SpecialCharTok{$}\NormalTok{place }\OtherTok{\textless{}{-}} \StringTok{"Sutton Bonnington"}
\NormalTok{suttonboningtondata}\SpecialCharTok{$}\NormalTok{lattitude }\OtherTok{\textless{}{-}} \FloatTok{52.833}
\NormalTok{suttonboningtondata}\SpecialCharTok{$}\NormalTok{logitude }\OtherTok{\textless{}{-}} \FloatTok{52.833}
\NormalTok{suttonboningtondata}\SpecialCharTok{$}\NormalTok{logitude }\OtherTok{\textless{}{-}} \SpecialCharTok{{-}}\FloatTok{1.250}
\NormalTok{suttonboningtondata}\SpecialCharTok{$}\NormalTok{easting }\OtherTok{\textless{}{-}} \DecValTok{450700}
\NormalTok{suttonboningtondata}\SpecialCharTok{$}\NormalTok{northing }\OtherTok{\textless{}{-}} \DecValTok{325900}
\NormalTok{suttonboningtondata}\SpecialCharTok{$}\NormalTok{height.amsl.metre }\OtherTok{\textless{}{-}} \DecValTok{48}
\end{Highlighting}
\end{Shaded}
\end{frame}

\begin{frame}[fragile]{Working With Datum Triples}
\protect\hypertarget{working-with-datum-triples}{}
The datum triple is the universal starting point for all data.
Converting to this format allows us to combine data from multiple
sources, as long as the \texttt{datumEntity} only ever refers to a
unique observation. Note that this format makes no presumptions about
attributes, but it does require all data to be represented as
characters. There is no reason to keep any attribute with a value of
\texttt{NA} as the row contains no information.

\begin{Shaded}
\begin{Highlighting}[]
\NormalTok{suttonbonington.triple }\OtherTok{\textless{}{-}}\NormalTok{ suttonboningtondata }\SpecialCharTok{\%\textgreater{}\%} 
  \FunctionTok{map\_df}\NormalTok{(as.character) }\SpecialCharTok{\%\textgreater{}\%} \CommentTok{\# Convert to character}
  \FunctionTok{pivot\_longer}\NormalTok{(}\SpecialCharTok{!}\NormalTok{datumEntity, }\CommentTok{\# datumEntity as  key}
  \AttributeTok{names\_to =} \StringTok{"datumAttribute"}\NormalTok{, }\CommentTok{\# Attribute name}
  \AttributeTok{values\_to =} \StringTok{"datumValue"} \CommentTok{\# Save  value}
\NormalTok{      ) }\SpecialCharTok{\%\textgreater{}\%} \FunctionTok{drop\_na}\NormalTok{() }\CommentTok{\# Drop NAs}
\end{Highlighting}
\end{Shaded}
\end{frame}

\begin{frame}[fragile]{Saving The Raw Data}
\protect\hypertarget{saving-the-raw-data}{}
Finally we can save the data in a machine readable format for reuse.
Although we have termed this as \texttt{data-raw}, multiple choices have
been made in its transformation into this state. It should be clear that
this \texttt{state} makes no assumptions about numbers of fields in any
observation.

\begin{Shaded}
\begin{Highlighting}[]
\FunctionTok{dir.create}\NormalTok{( }
\NormalTok{          fs}\SpecialCharTok{::}\FunctionTok{path}\NormalTok{( absolute.path,}
            \StringTok{"data{-}raw"}\NormalTok{ ),  }
            \AttributeTok{showWarnings =} \ConstantTok{FALSE}\NormalTok{,}
            \AttributeTok{recursive =} \ConstantTok{TRUE}\NormalTok{) }\CommentTok{\# }

\NormalTok{path.to.save}\OtherTok{\textless{}{-}}\NormalTok{ fs}\SpecialCharTok{::}\FunctionTok{path}\NormalTok{( absolute.path,}
 \StringTok{"data{-}raw"}\NormalTok{,}\StringTok{"suttonboningtondata"}\NormalTok{, }\AttributeTok{ext =} \StringTok{"rds"}\NormalTok{)}
\FunctionTok{saveRDS}\NormalTok{(suttonbonington.triple,path.to.save)}
\end{Highlighting}
\end{Shaded}
\end{frame}

\begin{frame}[fragile]{Starting The Analysis With Data-Raw}
\protect\hypertarget{starting-the-analysis-with-data-raw}{}
For the purpose of this vignette we load the \texttt{data-raw} to
demonstrate the the analysis could start with multiple files using the
search style loading.

\begin{Shaded}
\begin{Highlighting}[]
\CommentTok{\# Load data{-}raw}
\NormalTok{path.to.data }\OtherTok{\textless{}{-}}\NormalTok{ fs}\SpecialCharTok{::}\FunctionTok{path}\NormalTok{( absolute.path}
\NormalTok{                               ,}\StringTok{"data{-}raw"}\NormalTok{)}
\NormalTok{weather.data.triple }\OtherTok{\textless{}{-}} \FunctionTok{list.files}\NormalTok{(}
\NormalTok{  path.to.data,}
  \AttributeTok{pattern =} \StringTok{".rds$"}\NormalTok{, }\CommentTok{\# Make a suitable filter.}
  \AttributeTok{full.names =} \ConstantTok{TRUE}\NormalTok{,}
  \AttributeTok{recursive =} \ConstantTok{TRUE}\NormalTok{) }\SpecialCharTok{\%\textgreater{}\%}
\NormalTok{   purrr}\SpecialCharTok{::}\FunctionTok{map\_df}\NormalTok{(readRDS) }\SpecialCharTok{\%\textgreater{}\%}
   \FunctionTok{rbind}\NormalTok{()}
\end{Highlighting}
\end{Shaded}
\end{frame}

\begin{frame}[fragile]{File Contents}
\protect\hypertarget{file-contents}{}
The first 13 rows of the \texttt{weather.data.triple} now look like
this:

\begin{longtable}[]{@{}
  >{\centering\arraybackslash}p{(\columnwidth - 4\tabcolsep) * \real{0.39}}
  >{\centering\arraybackslash}p{(\columnwidth - 4\tabcolsep) * \real{0.28}}
  >{\centering\arraybackslash}p{(\columnwidth - 4\tabcolsep) * \real{0.28}}@{}}
\toprule
\begin{minipage}[b]{\linewidth}\centering
datumEntity
\end{minipage} & \begin{minipage}[b]{\linewidth}\centering
datumAttribute
\end{minipage} & \begin{minipage}[b]{\linewidth}\centering
datumValue
\end{minipage} \\
\midrule
\endhead
Sutton.Bonnington:1959:01 & YYYY & 1959 \\
Sutton.Bonnington:1959:01 & mm & 01 \\
Sutton.Bonnington:1959:01 & tmax.degC & 4.2 \\
Sutton.Bonnington:1959:01 & tmin.degC & -2.4 \\
Sutton.Bonnington:1959:01 & airfrost.days & 23 \\
Sutton.Bonnington:1959:01 & sun.hours & 78.8 \\
Sutton.Bonnington:1959:01 & YYYYMMDD & 1959-01-31 \\
Sutton.Bonnington:1959:01 & place & Sutton Bonnington \\
Sutton.Bonnington:1959:01 & lattitude & 52.833 \\
Sutton.Bonnington:1959:01 & logitude & -1.25 \\
Sutton.Bonnington:1959:01 & easting & 450700 \\
Sutton.Bonnington:1959:01 & northing & 325900 \\
Sutton.Bonnington:1959:01 & height.amsl.metre & 48 \\
\bottomrule
\end{longtable}
\end{frame}

\begin{frame}[fragile]{Prepare \texttt{data-sl}}
\protect\hypertarget{prepare-data-sl}{}
From the raw data we now prepare \texttt{data-sl} which is a loosely
defined format of convenience. All the attributes are in character
format, but may be in many different units. Re-arranging into a wider
format is helpful

\begin{Shaded}
\begin{Highlighting}[]
\CommentTok{\# Prepare data{-}sl}
\CommentTok{\# First prepare the long data}
\NormalTok{VC55.weather.sl }\OtherTok{\textless{}{-}}\NormalTok{ weather.data.triple }\SpecialCharTok{\%\textgreater{}\%}
                  \FunctionTok{pivot\_wider}\NormalTok{(}
  \AttributeTok{id\_cols =}\NormalTok{ datumEntity,}
  \AttributeTok{names\_from =}\NormalTok{ datumAttribute,}
  \AttributeTok{values\_from =}\NormalTok{ datumValue)}
\end{Highlighting}
\end{Shaded}
\end{frame}

\begin{frame}[fragile]{Why Use Wide Format?}
\protect\hypertarget{why-use-wide-format}{}
To prepare strictly defined data to analyse the weather we select the
columns of interest and make a long format with the date as the key
column. At this point we loose any columns that are not required that
may have be present if multiple sources of \texttt{data-raw} were used.

We choose this format as it gives great flexibility with analysis and
works well with a Grammar of Graphic (GoG) approach. This contrasts with
the temptation to produce a wide format of data as one might use in a
spreadsheet analysis. The versatility of GoG will become apparent as we
proceed and see how all plots may be specified by changing the GoG
verbs.
\end{frame}

\begin{frame}[fragile]{Create Data-ss}
\protect\hypertarget{create-data-ss}{}
\begin{Shaded}
\begin{Highlighting}[]
\CommentTok{\# But we are interesting in plotting the data by date so}

\NormalTok{  VC55.weather.ss }\OtherTok{\textless{}{-}}\NormalTok{ VC55.weather.sl }\SpecialCharTok{\%\textgreater{}\%} 
    \FunctionTok{select}\NormalTok{(}\StringTok{"YYYYMMDD"}\NormalTok{, }
           \StringTok{"tmax.degC"}\NormalTok{,}
           \StringTok{"tmin.degC"}\NormalTok{,}
           \StringTok{"airfrost.days"}\NormalTok{, }
           \StringTok{"sun.hours"}\NormalTok{ ,}
           \StringTok{"rain.mm"}\NormalTok{)     }\SpecialCharTok{\%\textgreater{}\%}
            \FunctionTok{pivot\_longer}\NormalTok{(}\SpecialCharTok{!}\NormalTok{YYYYMMDD,}
                  \AttributeTok{names\_to =} \StringTok{"datumAttribute"}\NormalTok{, }
                  \AttributeTok{values\_to =} \StringTok{"datumValue"}\NormalTok{) }
\end{Highlighting}
\end{Shaded}
\end{frame}

\begin{frame}[fragile]{Specify Data types}
\protect\hypertarget{specify-data-types}{}
At this point we can specify data types as Date and numeric for
analysis.

\begin{Shaded}
\begin{Highlighting}[]
\CommentTok{\# Specify data types}
\NormalTok{VC55.weather.ss}\SpecialCharTok{$}\NormalTok{YYYYMMDD }\OtherTok{\textless{}{-}}\NormalTok{  VC55.weather.ss}\SpecialCharTok{$}\NormalTok{YYYYMMDD }\SpecialCharTok{\%\textgreater{}\%}\NormalTok{ as.Date}
\NormalTok{VC55.weather.ss}\SpecialCharTok{$}\NormalTok{datumValue }\OtherTok{\textless{}{-}}\NormalTok{ VC55.weather.ss}\SpecialCharTok{$}\NormalTok{datumValue }\SpecialCharTok{\%\textgreater{}\%} \FunctionTok{as.numeric}\NormalTok{()}
\end{Highlighting}
\end{Shaded}
\end{frame}

\begin{frame}[fragile]{Save Data-ss}
\protect\hypertarget{save-data-ss}{}
We can now save into \texttt{data-ss}.

\begin{Shaded}
\begin{Highlighting}[]
\FunctionTok{dir.create}\NormalTok{( }
\NormalTok{          fs}\SpecialCharTok{::}\FunctionTok{path}\NormalTok{( absolute.path,}
            \StringTok{"data{-}ss"}\NormalTok{ ),  }
            \AttributeTok{showWarnings =} \ConstantTok{FALSE}\NormalTok{,}
            \AttributeTok{recursive =} \ConstantTok{TRUE}\NormalTok{) }\CommentTok{\# Create the directory}

\CommentTok{\#And save.}
\NormalTok{path.to.save.ss}\OtherTok{\textless{}{-}}\NormalTok{ fs}\SpecialCharTok{::}\FunctionTok{path}\NormalTok{( absolute.path}
\NormalTok{                               ,}\StringTok{"data{-}ss"}\NormalTok{,}\StringTok{"VC55.weather.ss"}\NormalTok{, }\AttributeTok{ext =} \StringTok{"rds"}\NormalTok{)}
\FunctionTok{saveRDS}\NormalTok{(VC55.weather.ss, path.to.save.ss)}
\end{Highlighting}
\end{Shaded}
\end{frame}

\begin{frame}[fragile]{Why Save Data-ss?}
\protect\hypertarget{why-save-data-ss}{}
Once again the data may be loaded and the analysis start from this
point. Rather than load as a named file, we demonstrate how multiple
files in \texttt{data-ss} may be loaded and combined in a simple action.
Since \texttt{data-ss} are strictly defined each file may be `stacked'
to combine into a larger data-set. If this were really the case then it
would make sense to rename the data to something more appropriate. The
named method of loading is included as comments for comparison.
\end{frame}

\begin{frame}[fragile]{Load Data-ss}
\protect\hypertarget{load-data-ss}{}
\begin{Shaded}
\begin{Highlighting}[]
\CommentTok{\# Load data{-}ss}
\NormalTok{path.to.data.ss }\OtherTok{\textless{}{-}}\NormalTok{ fs}\SpecialCharTok{::}\FunctionTok{path}\NormalTok{(absolute.path,}\StringTok{"data{-}ss"}\NormalTok{)}

\NormalTok{VC55.weather.ss }\OtherTok{\textless{}{-}} \FunctionTok{list.files}\NormalTok{(}
\NormalTok{  path.to.data.ss,}
  \AttributeTok{pattern =} \StringTok{".rds"}\NormalTok{, }\CommentTok{\# Make a suitable filter. Use the dot for a wildcard.}
  \AttributeTok{full.names =} \ConstantTok{TRUE}\NormalTok{,}
  \AttributeTok{recursive =} \ConstantTok{TRUE}\NormalTok{)  }\SpecialCharTok{\%\textgreater{}\%}
\NormalTok{  purrr}\SpecialCharTok{::}\FunctionTok{map\_df}\NormalTok{(readRDS) }
\end{Highlighting}
\end{Shaded}
\end{frame}

\begin{frame}[fragile]{A Simple Check Plot}
\protect\hypertarget{a-simple-check-plot}{}
By faceting on \texttt{datumAttribute} we can produce a separate graph
for each attribute. While is shows we have data, each graph has
different units, so the scales do not make sense.

\begin{Shaded}
\begin{Highlighting}[]
\NormalTok{data.to.plot }\OtherTok{\textless{}{-}}\NormalTok{ VC55.weather.ss}
\NormalTok{data.to.plot }\SpecialCharTok{\%\textgreater{}\%} \FunctionTok{drop\_na}\NormalTok{() }\SpecialCharTok{\%\textgreater{}\%}
\FunctionTok{ggplot}\NormalTok{(}\FunctionTok{aes}\NormalTok{(}\AttributeTok{colour =}\NormalTok{ datumAttribute, }
           \AttributeTok{x=}\NormalTok{ YYYYMMDD, }
           \AttributeTok{y =}\NormalTok{datumValue, }
           \AttributeTok{group =}\NormalTok{ datumAttribute) ) }\SpecialCharTok{+} 
 \CommentTok{\# geom\_point() +}
  \FunctionTok{geom\_smooth}\NormalTok{( }\AttributeTok{se =} \ConstantTok{FALSE}\NormalTok{, }
     \AttributeTok{method =} \StringTok{"loess"}\NormalTok{, }\AttributeTok{formula =} \StringTok{"y \textasciitilde{} x"}\NormalTok{) }\SpecialCharTok{+}
  \FunctionTok{facet\_wrap}\NormalTok{(}\SpecialCharTok{\textasciitilde{}}\NormalTok{ datumAttribute)}
\end{Highlighting}
\end{Shaded}
\end{frame}

\begin{frame}[fragile]{Simple Check Plot}
\protect\hypertarget{simple-check-plot}{}
By faceting on \texttt{datumAttribute} we can produce a separate graph
for each attribute. While is shows we have data, each graph has
different units, so the scales do not make sense.

\begin{center}\includegraphics[width=0.7\linewidth]{Sutton_Bonnington_weather_files/figure-beamer/SimpleCheckPlot-1} \end{center}
\end{frame}

\begin{frame}[fragile]{Normalising With Z-scores}
\protect\hypertarget{normalising-with-z-scores}{}
However, if we use z-scores instead then each plot is normalised against
zero and the standard deviation.

\begin{Shaded}
\begin{Highlighting}[]
\NormalTok{data.to.plot }\OtherTok{\textless{}{-}}\NormalTok{ VC55.weather.ss}
\NormalTok{ data.to.plot }\SpecialCharTok{\%\textgreater{}\%} \FunctionTok{drop\_na}\NormalTok{() }\SpecialCharTok{\%\textgreater{}\%}
  \FunctionTok{group\_by}\NormalTok{(datumAttribute) }\SpecialCharTok{\%\textgreater{}\%} 
  \FunctionTok{mutate}\NormalTok{( }\AttributeTok{value =}\NormalTok{ datumValue,}
  \AttributeTok{Z.score =} 
\NormalTok{    (datumValue }\SpecialCharTok{{-}}\FunctionTok{mean}\NormalTok{(datumValue))}\SpecialCharTok{/}\FunctionTok{sd}\NormalTok{(datumValue)}
\NormalTok{        ) }\SpecialCharTok{\%\textgreater{}\%}
  \FunctionTok{ggplot}\NormalTok{( }\FunctionTok{aes}\NormalTok{(}\AttributeTok{colour =}\NormalTok{ datumAttribute, }
              \AttributeTok{x=}\NormalTok{ YYYYMMDD, }\AttributeTok{y =}\NormalTok{Z.score, }
              \AttributeTok{group =}\NormalTok{ datumAttribute) ) }\SpecialCharTok{+} 
              \FunctionTok{geom\_smooth}\NormalTok{( }\AttributeTok{se =} \ConstantTok{FALSE}\NormalTok{, }
                           \AttributeTok{method =} \StringTok{"loess"}\NormalTok{, }
                           \AttributeTok{formula =} \StringTok{"y \textasciitilde{} x"}\NormalTok{) }\SpecialCharTok{+} 
              \FunctionTok{scale\_colour\_viridis\_d}\NormalTok{()}
\end{Highlighting}
\end{Shaded}
\end{frame}

\begin{frame}{Plotting With Z-scores}
\protect\hypertarget{plotting-with-z-scores}{}
Rather than 5 separate plots, we can use a single graph to plot a smooth
loess regression for each attribute.

\begin{center}\includegraphics[width=0.7\linewidth]{Sutton_Bonnington_weather_files/figure-beamer/NormalisingWithZScore-1} \end{center}
\end{frame}

\begin{frame}[fragile]{Checking Data Distribution}
\protect\hypertarget{checking-data-distribution}{}
\begin{Shaded}
\begin{Highlighting}[]
\NormalTok{my.dist }\OtherTok{\textless{}{-}} \StringTok{"norm"}
\NormalTok{data.to.plot }\OtherTok{\textless{}{-}}\NormalTok{ VC55.weather.ss}
\NormalTok{ data.to.plot }\SpecialCharTok{\%\textgreater{}\%} \FunctionTok{drop\_na}\NormalTok{() }\SpecialCharTok{\%\textgreater{}\%}
  \FunctionTok{group\_by}\NormalTok{(datumAttribute) }\SpecialCharTok{\%\textgreater{}\%} 
  \FunctionTok{mutate}\NormalTok{( }\AttributeTok{value =}\NormalTok{ datumValue,}
      \AttributeTok{Z.score =} 
\NormalTok{        (datumValue }\SpecialCharTok{{-}}\FunctionTok{mean}\NormalTok{(datumValue))}\SpecialCharTok{/}\FunctionTok{sd}\NormalTok{(datumValue)}
\NormalTok{        ) }\SpecialCharTok{\%\textgreater{}\%}
  \FunctionTok{ggplot}\NormalTok{(}\AttributeTok{mapping =} \FunctionTok{aes}\NormalTok{( }
                      \AttributeTok{group =}\NormalTok{ datumAttribute,}
                      \AttributeTok{sample =}\NormalTok{ Z.score)) }\SpecialCharTok{+}
    \FunctionTok{stat\_qq\_line}\NormalTok{(}\AttributeTok{distribution =}\NormalTok{ my.dist) }\SpecialCharTok{+}
   \FunctionTok{stat\_qq\_point}\NormalTok{(}\AttributeTok{distribution =}\NormalTok{ my.dist) }\SpecialCharTok{+} 
  \FunctionTok{scale\_colour\_viridis\_d}\NormalTok{() }\SpecialCharTok{+}
  \FunctionTok{ggtitle}\NormalTok{(}\StringTok{"Z{-}score compared to normal distribution"}\NormalTok{) }\SpecialCharTok{+} 
   \FunctionTok{facet\_wrap}\NormalTok{(}\SpecialCharTok{\textasciitilde{}}\NormalTok{ datumAttribute)}
\end{Highlighting}
\end{Shaded}
\end{frame}

\begin{frame}{Q-Q Plot Z-score Compared To Normal Distribution}
\protect\hypertarget{q-q-plot-z-score-compared-to-normal-distribution}{}
This looks good but it is predicated upon the deviations being normally
distributed about the mean.

\begin{center}\includegraphics[width=0.7\linewidth]{Sutton_Bonnington_weather_files/figure-beamer/CheckingDataDistributionQQ-1} \end{center}
\end{frame}

\begin{frame}{Assumptions}
\protect\hypertarget{assumptions}{}
Our assumption or a normal distribution is reasonably valid, but not
perfect. As we are looking at weather data, it might be nice to look at:
Annual Total Rainfall, Annual Air Frost days, Annual Maximum
Temperature, and Annual Minimum Temperature. Again GoG comes to the
rescue and we can quickly produce the following plots.
\end{frame}

\begin{frame}[fragile]{Weather Plots In Native Units}
\protect\hypertarget{weather-plots-in-native-units}{}
For convenience, the plot routine has been written as a function so we
can get all the plots quickly by reusing the code and just changing the
selection attribute.

\begin{Shaded}
\begin{Highlighting}[]
\NormalTok{ggfun }\OtherTok{\textless{}{-}} \ControlFlowTok{function}\NormalTok{(dat, nice.title)\{}
\NormalTok{  plot.output }\OtherTok{\textless{}{-}} \FunctionTok{ggplot}\NormalTok{(}\AttributeTok{data =}\NormalTok{ dat,}
                \FunctionTok{aes}\NormalTok{(}\AttributeTok{x =}\NormalTok{ YYYY,}
                    \AttributeTok{y =}\NormalTok{ annual)) }\SpecialCharTok{+}
    \FunctionTok{geom\_point}\NormalTok{() }\SpecialCharTok{+}
    \FunctionTok{geom\_smooth}\NormalTok{( }\AttributeTok{se =} \ConstantTok{FALSE}\NormalTok{, }
             \AttributeTok{method =} \StringTok{"lm"}\NormalTok{, }\AttributeTok{formula =} \StringTok{"y \textasciitilde{} x"}\NormalTok{) }\SpecialCharTok{+}
  \FunctionTok{geom\_line}\NormalTok{()}\SpecialCharTok{+} \FunctionTok{ggtitle}\NormalTok{( nice.title ) }\SpecialCharTok{+} \FunctionTok{ylab}\NormalTok{(nice.title)}
  \FunctionTok{return}\NormalTok{(plot.output)}
\NormalTok{\}}
\end{Highlighting}
\end{Shaded}
\end{frame}

\begin{frame}[fragile]{Filter The Data}
\protect\hypertarget{filter-the-data}{}
\begin{Shaded}
\begin{Highlighting}[]
\CommentTok{\# Weather "tmax.degC"     "tmin.degC"    }
\CommentTok{\# "airfrost.days" "sun.hours"     "rain.mm"}
\NormalTok{data.to.plot.local }\OtherTok{\textless{}{-}}\NormalTok{ VC55.weather.ss[}
\NormalTok{  VC55.weather.ss}\SpecialCharTok{$}\NormalTok{datumAttribute }\SpecialCharTok{==} \StringTok{"airfrost.days"}\NormalTok{, ]}
\NormalTok{data.to.plot.local}\SpecialCharTok{$}\NormalTok{YYYY }\OtherTok{\textless{}{-}} \FunctionTok{format}\NormalTok{(}
\NormalTok{  data.to.plot.local}\SpecialCharTok{$}\NormalTok{YYYYMMDD, }\AttributeTok{format =} \StringTok{"\%Y"}\NormalTok{) }\SpecialCharTok{\%\textgreater{}\%} 
    \FunctionTok{as.numeric}\NormalTok{()}
\NormalTok{data.to.plot.annual }\OtherTok{\textless{}{-}}\NormalTok{ data.to.plot.local }\SpecialCharTok{\%\textgreater{}\%} 
  \FunctionTok{drop\_na}\NormalTok{() }\SpecialCharTok{\%\textgreater{}\%} \FunctionTok{group\_by}\NormalTok{(YYYY) }\SpecialCharTok{\%\textgreater{}\%}
    \FunctionTok{mutate}\NormalTok{( }
  \CommentTok{\# Change to max, min etc. as required.}
          \AttributeTok{annual =} \FunctionTok{sum}\NormalTok{(datumValue)}
\NormalTok{        ) }
\end{Highlighting}
\end{Shaded}
\end{frame}

\begin{frame}{Plot Total Airfrost Days}
\protect\hypertarget{plot-total-airfrost-days}{}
\begin{center}\includegraphics[width=0.7\linewidth]{Sutton_Bonnington_weather_files/figure-beamer/PlotAirFrost-1} \end{center}
\end{frame}

\begin{frame}{Plot Maximum Temperature}
\protect\hypertarget{plot-maximum-temperature}{}
\begin{center}\includegraphics[width=0.7\linewidth]{Sutton_Bonnington_weather_files/figure-beamer/PlotMaxTemp-1} \end{center}
\end{frame}

\begin{frame}{Plot Minimum Temperature}
\protect\hypertarget{plot-minimum-temperature}{}
\begin{center}\includegraphics[width=0.7\linewidth]{Sutton_Bonnington_weather_files/figure-beamer/PlotMinTemp-1} \end{center}
\end{frame}

\begin{frame}{Plot Total Rainfail}
\protect\hypertarget{plot-total-rainfail}{}
\begin{center}\includegraphics[width=0.7\linewidth]{Sutton_Bonnington_weather_files/figure-beamer/PlotTotalRainfall-1} \end{center}
\end{frame}

\begin{frame}{Conclusion}
\protect\hypertarget{conclusion}{}
This vignette demonstrates the versatility of using data state in
conjunction with GoG.
\end{frame}

\end{document}
