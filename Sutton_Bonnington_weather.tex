\documentclass{article}

\usepackage{arxiv}

\usepackage[utf8]{inputenc} % allow utf-8 input
\usepackage[T1]{fontenc}    % use 8-bit T1 fonts
\usepackage{lmodern}        % https://github.com/rstudio/rticles/issues/343
\usepackage{hyperref}       % hyperlinks
\usepackage{url}            % simple URL typesetting
\usepackage{booktabs}       % professional-quality tables
\usepackage{amsfonts}       % blackboard math symbols
\usepackage{nicefrac}       % compact symbols for 1/2, etc.
\usepackage{microtype}      % microtypography
\usepackage{lipsum}
\usepackage{graphicx}

\title{\myTitleMainTitle}

\author{
    Dr Paul J. Palmer
   \\
    Wolfson School of Mechanical, Electrical and Manufacturing Engineering \\
    Loughborough University \\
  Leicestershire VC55 \\
  \texttt{\href{mailto:p.j.palmer@lboro.ac.uk}{\nolinkurl{p.j.palmer@lboro.ac.uk}}} \\
  }

\usepackage{color}
\usepackage{fancyvrb}
\newcommand{\VerbBar}{|}
\newcommand{\VERB}{\Verb[commandchars=\\\{\}]}
\DefineVerbatimEnvironment{Highlighting}{Verbatim}{commandchars=\\\{\}}
% Add ',fontsize=\small' for more characters per line
\usepackage{framed}
\definecolor{shadecolor}{RGB}{248,248,248}
\newenvironment{Shaded}{\begin{snugshade}}{\end{snugshade}}
\newcommand{\AlertTok}[1]{\textcolor[rgb]{0.94,0.16,0.16}{#1}}
\newcommand{\AnnotationTok}[1]{\textcolor[rgb]{0.56,0.35,0.01}{\textbf{\textit{#1}}}}
\newcommand{\AttributeTok}[1]{\textcolor[rgb]{0.77,0.63,0.00}{#1}}
\newcommand{\BaseNTok}[1]{\textcolor[rgb]{0.00,0.00,0.81}{#1}}
\newcommand{\BuiltInTok}[1]{#1}
\newcommand{\CharTok}[1]{\textcolor[rgb]{0.31,0.60,0.02}{#1}}
\newcommand{\CommentTok}[1]{\textcolor[rgb]{0.56,0.35,0.01}{\textit{#1}}}
\newcommand{\CommentVarTok}[1]{\textcolor[rgb]{0.56,0.35,0.01}{\textbf{\textit{#1}}}}
\newcommand{\ConstantTok}[1]{\textcolor[rgb]{0.00,0.00,0.00}{#1}}
\newcommand{\ControlFlowTok}[1]{\textcolor[rgb]{0.13,0.29,0.53}{\textbf{#1}}}
\newcommand{\DataTypeTok}[1]{\textcolor[rgb]{0.13,0.29,0.53}{#1}}
\newcommand{\DecValTok}[1]{\textcolor[rgb]{0.00,0.00,0.81}{#1}}
\newcommand{\DocumentationTok}[1]{\textcolor[rgb]{0.56,0.35,0.01}{\textbf{\textit{#1}}}}
\newcommand{\ErrorTok}[1]{\textcolor[rgb]{0.64,0.00,0.00}{\textbf{#1}}}
\newcommand{\ExtensionTok}[1]{#1}
\newcommand{\FloatTok}[1]{\textcolor[rgb]{0.00,0.00,0.81}{#1}}
\newcommand{\FunctionTok}[1]{\textcolor[rgb]{0.00,0.00,0.00}{#1}}
\newcommand{\ImportTok}[1]{#1}
\newcommand{\InformationTok}[1]{\textcolor[rgb]{0.56,0.35,0.01}{\textbf{\textit{#1}}}}
\newcommand{\KeywordTok}[1]{\textcolor[rgb]{0.13,0.29,0.53}{\textbf{#1}}}
\newcommand{\NormalTok}[1]{#1}
\newcommand{\OperatorTok}[1]{\textcolor[rgb]{0.81,0.36,0.00}{\textbf{#1}}}
\newcommand{\OtherTok}[1]{\textcolor[rgb]{0.56,0.35,0.01}{#1}}
\newcommand{\PreprocessorTok}[1]{\textcolor[rgb]{0.56,0.35,0.01}{\textit{#1}}}
\newcommand{\RegionMarkerTok}[1]{#1}
\newcommand{\SpecialCharTok}[1]{\textcolor[rgb]{0.00,0.00,0.00}{#1}}
\newcommand{\SpecialStringTok}[1]{\textcolor[rgb]{0.31,0.60,0.02}{#1}}
\newcommand{\StringTok}[1]{\textcolor[rgb]{0.31,0.60,0.02}{#1}}
\newcommand{\VariableTok}[1]{\textcolor[rgb]{0.00,0.00,0.00}{#1}}
\newcommand{\VerbatimStringTok}[1]{\textcolor[rgb]{0.31,0.60,0.02}{#1}}
\newcommand{\WarningTok}[1]{\textcolor[rgb]{0.56,0.35,0.01}{\textbf{\textit{#1}}}}

% Pandoc citation processing
\newlength{\csllabelwidth}
\setlength{\csllabelwidth}{3em}
\newlength{\cslhangindent}
\setlength{\cslhangindent}{1.5em}
% for Pandoc 2.8 to 2.10.1
\newenvironment{cslreferences}%
  {}%
  {\par}
% For Pandoc 2.11+
\newenvironment{CSLReferences}[2] % #1 hanging-ident, #2 entry spacing
 {% don't indent paragraphs
  \setlength{\parindent}{0pt}
  % turn on hanging indent if param 1 is 1
  \ifodd #1 \everypar{\setlength{\hangindent}{\cslhangindent}}\ignorespaces\fi
  % set entry spacing
  \ifnum #2 > 0
  \setlength{\parskip}{#2\baselineskip}
  \fi
 }%
 {}
\usepackage{calc} % for calculating minipage widths
\newcommand{\CSLBlock}[1]{#1\hfill\break}
\newcommand{\CSLLeftMargin}[1]{\parbox[t]{\csllabelwidth}{#1}}
\newcommand{\CSLRightInline}[1]{\parbox[t]{\linewidth - \csllabelwidth}{#1}\break}
\newcommand{\CSLIndent}[1]{\hspace{\cslhangindent}#1}

%\usepackage{booktabs}
%\usepackage{longtable}
%\usepackage{morefloats}
%\extrafloats{100}
%\date{August 25, 2021}
%\renewcommand{\today}{September 5, 2021}

% Set the copyright footer
%\lfoot{\copyright 2021 P.J. Palmer  P.M. Leonard}

% Some figure placement options.
%\usepackage[figuresonly,nomarkers,fighead, figlist]{endfloat}
%\usepackage[figuresonly,nomarkers,nolists]{endfloat}
% Put multiple figures per page
%\renewcommand{\efloatseparator}{\mbox{}}
\usepackage{flafter}
\usepackage{booktabs}
\usepackage{longtable}
\usepackage{array}
\usepackage{multirow}
\usepackage{wrapfig}
\usepackage{float}
\usepackage{colortbl}
\usepackage{pdflscape}
\usepackage{tabu}
\usepackage{threeparttable}
\usepackage{threeparttablex}
\usepackage[normalem]{ulem}
\usepackage{makecell}
\usepackage{xcolor}


\begin{document}
\maketitle

\def\tightlist{}


\begin{abstract}
\myAbstract
\end{abstract}


\hypertarget{introduction}{%
\section{Introduction}\label{introduction}}

\myTitleMainTitle

\hypertarget{read-sutton-bonnington-data}{%
\section{Read Sutton Bonnington Data}\label{read-sutton-bonnington-data}}

Load the Tidyverse libraries and other helpers.

\begin{Shaded}
\begin{Highlighting}[]
\CommentTok{\# Load libraries}
\FunctionTok{library}\NormalTok{(plyr)}
\FunctionTok{library}\NormalTok{(tidyverse)}
\FunctionTok{library}\NormalTok{(kableExtra)}
\FunctionTok{library}\NormalTok{(gridExtra)}
\FunctionTok{library}\NormalTok{(qqplotr)}
\FunctionTok{library}\NormalTok{(readr)}
\FunctionTok{library}\NormalTok{(timeDate)}
\end{Highlighting}
\end{Shaded}

Read the data. The absolute path helps to write code that is computer independent but it does use the Rstudio specific function to find the current project location. The use of \texttt{fs} to build the file path ensures compatibility across all operating systems. It also gives a chance to check that it is correct before loading the file.

\begin{Shaded}
\begin{Highlighting}[]
\CommentTok{\# Read data}
\NormalTok{absolute.path }\OtherTok{\textless{}{-}}\NormalTok{ rprojroot}\SpecialCharTok{::}\FunctionTok{find\_rstudio\_root\_file}\NormalTok{()}
\NormalTok{path.to.my.data }\OtherTok{\textless{}{-}}\NormalTok{ fs}\SpecialCharTok{::}\FunctionTok{path}\NormalTok{( absolute.path,}
                             \StringTok{"data{-}ext"}\NormalTok{,}
                             \StringTok{"suttonboningtondata"}\NormalTok{, }
                            \AttributeTok{ext =} \StringTok{"txt"}\NormalTok{)}
\end{Highlighting}
\end{Shaded}

The \texttt{read\_table} function does a good job of reading the data into columns recognising the partitioning with the use of spaces after the first five lines of text are skipped.

\begin{Shaded}
\begin{Highlighting}[]
\NormalTok{suttonboningtondata }\OtherTok{\textless{}{-}} \FunctionTok{read\_table}\NormalTok{(}
\NormalTok{                        path.to.my.data, }
                        \AttributeTok{skip =} \DecValTok{5}\NormalTok{) }\CommentTok{\# skip 5 lines of text}
\end{Highlighting}
\end{Shaded}

Renaming the column names requires checking the results manually with \texttt{view(suttonboningtondata)} befor making a list of names.

\begin{Shaded}
\begin{Highlighting}[]
\CommentTok{\# Rename the columns to something useful}
\NormalTok{new.colnames }\OtherTok{\textless{}{-}} \FunctionTok{c}\NormalTok{( }\StringTok{"YYYY"}\NormalTok{,}
                   \StringTok{"mm"}\NormalTok{,}
                   \StringTok{"tmax.degC"}\NormalTok{,}
                   \StringTok{"tmin.degC"}\NormalTok{,}
                   \StringTok{"airfrost.days"}\NormalTok{,}
                   \StringTok{"rain.mm"}\NormalTok{,}
                   \StringTok{"sun.hours"}
\NormalTok{  )}
\FunctionTok{colnames}\NormalTok{(suttonboningtondata) }\OtherTok{\textless{}{-}}\NormalTok{ new.colnames}
\end{Highlighting}
\end{Shaded}

The text file headings were split across two rows so the first row of the data contains fragments which we can drop since we have included them in our column names manually.

\begin{Shaded}
\begin{Highlighting}[]
\CommentTok{\# Drop row 1}
\NormalTok{suttonboningtondata }\OtherTok{\textless{}{-}}\NormalTok{ suttonboningtondata[}\SpecialCharTok{{-}}\DecValTok{1}\NormalTok{,]}
\end{Highlighting}
\end{Shaded}

We can now remove the unwanted characters that remain in the data. Note the use of the double escape \texttt{\textbackslash{}\textbackslash{}*} to select the asterisk. The removal of unwanted special characters is often a problem in nascent data so it is best to get rid of them before they cause problems when code is executed..

\begin{Shaded}
\begin{Highlighting}[]
\NormalTok{suttonboningtondata }\OtherTok{\textless{}{-}}\NormalTok{ suttonboningtondata }\SpecialCharTok{\%\textgreater{}\%}
                        \FunctionTok{map\_df}\NormalTok{( gsub,}
                                \AttributeTok{pattern =} \StringTok{"}\SpecialCharTok{\textbackslash{}\textbackslash{}}\StringTok{*"}\NormalTok{,}
                                \AttributeTok{replacement =} \StringTok{""}\NormalTok{) }\SpecialCharTok{\%\textgreater{}\%}
                        \FunctionTok{map\_df}\NormalTok{( gsub,}
                                \AttributeTok{pattern =} \StringTok{"{-}{-}{-}"}\NormalTok{,}
                                \AttributeTok{replacement =} \ConstantTok{NA}
\NormalTok{                                )}
\end{Highlighting}
\end{Shaded}

The early months of the year are represented by a single digit so we need to pad them with a leading zero. The we can create a date in the ISO recognised format: \texttt{YYYY-MM-DD}. As these are monthly summaries we choose a date near the end of the month which allows us to convert from a character string to a date when we plot results.

\begin{Shaded}
\begin{Highlighting}[]
\CommentTok{\# Date is not properly encoded so build a date from the data.}
\CommentTok{\# The weather is summaries monthly so we need the last day of the month.}
\NormalTok{suttonboningtondata}\SpecialCharTok{$}\NormalTok{mm }\OtherTok{\textless{}{-}}\NormalTok{   stringr}\SpecialCharTok{::}\FunctionTok{str\_pad}\NormalTok{(suttonboningtondata}\SpecialCharTok{$}\NormalTok{mm, }\AttributeTok{width =} \DecValTok{2}\NormalTok{, }\AttributeTok{pad =} \StringTok{"0"}\NormalTok{)}
 
 \CommentTok{\# Create a date at the end of the month.}
\NormalTok{dummyDate }\OtherTok{\textless{}{-}} \FunctionTok{with}\NormalTok{( suttonboningtondata, }
                  \FunctionTok{paste}\NormalTok{( YYYY, }
\NormalTok{                  mm, }
                  \StringTok{"28"}\NormalTok{, }\CommentTok{\# Need to put a day in every month.}
                  \AttributeTok{sep=} \StringTok{"{-}"}\NormalTok{   ) }
\NormalTok{                  )}
\CommentTok{\# Calculates the last day of the month}
\NormalTok{ suttonboningtondata}\SpecialCharTok{$}\NormalTok{YYYYMMDD }\OtherTok{\textless{}{-}} \FunctionTok{timeLastDayInMonth}\NormalTok{(dummyDate) }\SpecialCharTok{\%\textgreater{}\%} \FunctionTok{as.character}\NormalTok{()}
\end{Highlighting}
\end{Shaded}

Each row of data needs a unique identifier which we call \texttt{datumEntity}.

\begin{Shaded}
\begin{Highlighting}[]
\NormalTok{suttonboningtondata}\SpecialCharTok{$}\NormalTok{datumEntity }\OtherTok{\textless{}{-}} \FunctionTok{with}\NormalTok{(suttonboningtondata,}
                                   \FunctionTok{paste}\NormalTok{(}\StringTok{"Sutton.Bonnington"}\NormalTok{, YYYY, mm, }\AttributeTok{sep =} \StringTok{":"}\NormalTok{ )}
\NormalTok{                                   )}
\end{Highlighting}
\end{Shaded}

There are also some data attributes which were described in the five lines of text which apply to all rows of data. We manually add them here as they will be useful if we combine our data with similar sources from other locations.

\begin{Shaded}
\begin{Highlighting}[]
\CommentTok{\# manually name some values}
\NormalTok{suttonboningtondata}\SpecialCharTok{$}\NormalTok{place }\OtherTok{\textless{}{-}} \StringTok{"Sutton Bonnington"}
\NormalTok{suttonboningtondata}\SpecialCharTok{$}\NormalTok{lattitude }\OtherTok{\textless{}{-}} \FloatTok{52.833}
\NormalTok{suttonboningtondata}\SpecialCharTok{$}\NormalTok{logitude }\OtherTok{\textless{}{-}} \FloatTok{52.833}
\NormalTok{suttonboningtondata}\SpecialCharTok{$}\NormalTok{logitude }\OtherTok{\textless{}{-}} \SpecialCharTok{{-}}\FloatTok{1.250}
\NormalTok{suttonboningtondata}\SpecialCharTok{$}\NormalTok{easting }\OtherTok{\textless{}{-}} \DecValTok{450700}
\NormalTok{suttonboningtondata}\SpecialCharTok{$}\NormalTok{northing }\OtherTok{\textless{}{-}} \DecValTok{325900}
\NormalTok{suttonboningtondata}\SpecialCharTok{$}\NormalTok{height.amsl.metre }\OtherTok{\textless{}{-}} \DecValTok{48}
\end{Highlighting}
\end{Shaded}

The datum triple is the universal starting point for all data. Converting to this format allows us to combine data from multiple sources, as long as the \texttt{datumEntity} only ever refers to a unique observation. Note that this format makes no presumptions about attributes, but it does require all data to be represented as characters. There is no reason to keep any attribute with a value of \texttt{NA} as the row contains no information.

\begin{Shaded}
\begin{Highlighting}[]
\NormalTok{suttonbonington.triple }\OtherTok{\textless{}{-}}\NormalTok{ suttonboningtondata }\SpecialCharTok{\%\textgreater{}\%} 
                  \FunctionTok{map\_df}\NormalTok{(as.character) }\SpecialCharTok{\%\textgreater{}\%} \CommentTok{\# Convert everything to character}
                  \FunctionTok{pivot\_longer}\NormalTok{(}\SpecialCharTok{!}\NormalTok{datumEntity, }\CommentTok{\# Use datumEntity as the key}
                  \AttributeTok{names\_to =} \StringTok{"datumAttribute"}\NormalTok{, }\CommentTok{\# Save the name of the attribute}
                  \AttributeTok{values\_to =} \StringTok{"datumValue"} \CommentTok{\# Save the name of the value}
\NormalTok{                  ) }\SpecialCharTok{\%\textgreater{}\%} 
                  \FunctionTok{drop\_na}\NormalTok{() }\CommentTok{\# Drop all the NAs from datumValue}
\end{Highlighting}
\end{Shaded}

Finally we can save the data in a machine readable format for reuse. Effectively this is \texttt{raw} data that we have transformed into a readable state, so it is placed in a directory \texttt{data-raw}.

\begin{Shaded}
\begin{Highlighting}[]
\FunctionTok{dir.create}\NormalTok{( }
\NormalTok{          fs}\SpecialCharTok{::}\FunctionTok{path}\NormalTok{( absolute.path,}
            \StringTok{"data{-}raw"}\NormalTok{ ),  }
            \AttributeTok{showWarnings =} \ConstantTok{FALSE}\NormalTok{,}
            \AttributeTok{recursive =} \ConstantTok{TRUE}\NormalTok{) }\CommentTok{\# }

\NormalTok{path.to.save}\OtherTok{\textless{}{-}}\NormalTok{ fs}\SpecialCharTok{::}\FunctionTok{path}\NormalTok{( absolute.path}
\NormalTok{                               ,}\StringTok{"data{-}raw"}\NormalTok{,}\StringTok{"suttonboningtondata"}\NormalTok{, }\AttributeTok{ext =} \StringTok{"rds"}\NormalTok{)}
\FunctionTok{saveRDS}\NormalTok{(suttonbonington.triple,path.to.save)}
\end{Highlighting}
\end{Shaded}

For the purpose of this vignette we load the \texttt{data-raw} to demonstrate the the analysis could start from this point.

\begin{Shaded}
\begin{Highlighting}[]
\CommentTok{\# Load data{-}raw}

\NormalTok{path.to.data }\OtherTok{\textless{}{-}}\NormalTok{ fs}\SpecialCharTok{::}\FunctionTok{path}\NormalTok{( absolute.path}
\NormalTok{                               ,}\StringTok{"data{-}raw"}\NormalTok{,}\StringTok{"suttonboningtondata"}\NormalTok{, }\AttributeTok{ext =} \StringTok{"rds"}\NormalTok{)}
\CommentTok{\# Could change the name of the data if we so wished.}
\NormalTok{suttonboningtondata }\OtherTok{\textless{}{-}}  \FunctionTok{readRDS}\NormalTok{(path.to.data)}
\end{Highlighting}
\end{Shaded}

The first 10 rows of the \texttt{suttonboningtondata} now look like this:

\begin{longtable}[]{@{}ccc@{}}
\toprule
\begin{minipage}[b]{(\columnwidth - 2\tabcolsep) * \real{0.39}}\centering
datumEntity\strut
\end{minipage} & \begin{minipage}[b]{(\columnwidth - 2\tabcolsep) * \real{0.24}}\centering
datumAttribute\strut
\end{minipage} & \begin{minipage}[b]{(\columnwidth - 2\tabcolsep) * \real{0.18}}\centering
datumValue\strut
\end{minipage}\tabularnewline
\midrule
\endhead
\begin{minipage}[t]{(\columnwidth - 2\tabcolsep) * \real{0.39}}\centering
Sutton.Bonnington:1959:01\strut
\end{minipage} & \begin{minipage}[t]{(\columnwidth - 2\tabcolsep) * \real{0.24}}\centering
YYYY\strut
\end{minipage} & \begin{minipage}[t]{(\columnwidth - 2\tabcolsep) * \real{0.18}}\centering
1959\strut
\end{minipage}\tabularnewline
\begin{minipage}[t]{(\columnwidth - 2\tabcolsep) * \real{0.39}}\centering
Sutton.Bonnington:1959:01\strut
\end{minipage} & \begin{minipage}[t]{(\columnwidth - 2\tabcolsep) * \real{0.24}}\centering
mm\strut
\end{minipage} & \begin{minipage}[t]{(\columnwidth - 2\tabcolsep) * \real{0.18}}\centering
01\strut
\end{minipage}\tabularnewline
\begin{minipage}[t]{(\columnwidth - 2\tabcolsep) * \real{0.39}}\centering
Sutton.Bonnington:1959:01\strut
\end{minipage} & \begin{minipage}[t]{(\columnwidth - 2\tabcolsep) * \real{0.24}}\centering
tmax.degC\strut
\end{minipage} & \begin{minipage}[t]{(\columnwidth - 2\tabcolsep) * \real{0.18}}\centering
4.2\strut
\end{minipage}\tabularnewline
\begin{minipage}[t]{(\columnwidth - 2\tabcolsep) * \real{0.39}}\centering
Sutton.Bonnington:1959:01\strut
\end{minipage} & \begin{minipage}[t]{(\columnwidth - 2\tabcolsep) * \real{0.24}}\centering
tmin.degC\strut
\end{minipage} & \begin{minipage}[t]{(\columnwidth - 2\tabcolsep) * \real{0.18}}\centering
-2.4\strut
\end{minipage}\tabularnewline
\begin{minipage}[t]{(\columnwidth - 2\tabcolsep) * \real{0.39}}\centering
Sutton.Bonnington:1959:01\strut
\end{minipage} & \begin{minipage}[t]{(\columnwidth - 2\tabcolsep) * \real{0.24}}\centering
airfrost.days\strut
\end{minipage} & \begin{minipage}[t]{(\columnwidth - 2\tabcolsep) * \real{0.18}}\centering
23\strut
\end{minipage}\tabularnewline
\begin{minipage}[t]{(\columnwidth - 2\tabcolsep) * \real{0.39}}\centering
Sutton.Bonnington:1959:01\strut
\end{minipage} & \begin{minipage}[t]{(\columnwidth - 2\tabcolsep) * \real{0.24}}\centering
sun.hours\strut
\end{minipage} & \begin{minipage}[t]{(\columnwidth - 2\tabcolsep) * \real{0.18}}\centering
78.8\strut
\end{minipage}\tabularnewline
\bottomrule
\end{longtable}

From the raw data we now prepare \texttt{data-sl} which is a loosely defined format of convenience. All the attributes are in character format, but may be in many different units. Re-arranging into a wider format is helpful

\begin{Shaded}
\begin{Highlighting}[]
\CommentTok{\# Prepare data{-}ss}
\CommentTok{\# First prepare the long data}
\NormalTok{VC55.weather.sl }\OtherTok{\textless{}{-}}\NormalTok{ suttonbonington.triple }\SpecialCharTok{\%\textgreater{}\%}
                  \FunctionTok{pivot\_wider}\NormalTok{(}
  \AttributeTok{id\_cols =}\NormalTok{ datumEntity,}
  \AttributeTok{names\_from =}\NormalTok{ datumAttribute,}
  \AttributeTok{values\_from =}\NormalTok{ datumValue)}
\end{Highlighting}
\end{Shaded}

The wider format \texttt{VC55.weather.sl} now looks like:

\begin{longtable}[]{@{}cccccc@{}}
\caption{Table continues below}\tabularnewline
\toprule
\begin{minipage}[b]{(\columnwidth - 5\tabcolsep) * \real{0.35}}\centering
datumEntity\strut
\end{minipage} & \begin{minipage}[b]{(\columnwidth - 5\tabcolsep) * \real{0.09}}\centering
YYYY\strut
\end{minipage} & \begin{minipage}[b]{(\columnwidth - 5\tabcolsep) * \real{0.06}}\centering
mm\strut
\end{minipage} & \begin{minipage}[b]{(\columnwidth - 5\tabcolsep) * \real{0.15}}\centering
tmax.degC\strut
\end{minipage} & \begin{minipage}[b]{(\columnwidth - 5\tabcolsep) * \real{0.15}}\centering
tmin.degC\strut
\end{minipage} & \begin{minipage}[b]{(\columnwidth - 5\tabcolsep) * \real{0.20}}\centering
airfrost.days\strut
\end{minipage}\tabularnewline
\midrule
\endfirsthead
\toprule
\begin{minipage}[b]{(\columnwidth - 5\tabcolsep) * \real{0.35}}\centering
datumEntity\strut
\end{minipage} & \begin{minipage}[b]{(\columnwidth - 5\tabcolsep) * \real{0.09}}\centering
YYYY\strut
\end{minipage} & \begin{minipage}[b]{(\columnwidth - 5\tabcolsep) * \real{0.06}}\centering
mm\strut
\end{minipage} & \begin{minipage}[b]{(\columnwidth - 5\tabcolsep) * \real{0.15}}\centering
tmax.degC\strut
\end{minipage} & \begin{minipage}[b]{(\columnwidth - 5\tabcolsep) * \real{0.15}}\centering
tmin.degC\strut
\end{minipage} & \begin{minipage}[b]{(\columnwidth - 5\tabcolsep) * \real{0.20}}\centering
airfrost.days\strut
\end{minipage}\tabularnewline
\midrule
\endhead
\begin{minipage}[t]{(\columnwidth - 5\tabcolsep) * \real{0.35}}\centering
Sutton.Bonnington:1959:01\strut
\end{minipage} & \begin{minipage}[t]{(\columnwidth - 5\tabcolsep) * \real{0.09}}\centering
1959\strut
\end{minipage} & \begin{minipage}[t]{(\columnwidth - 5\tabcolsep) * \real{0.06}}\centering
01\strut
\end{minipage} & \begin{minipage}[t]{(\columnwidth - 5\tabcolsep) * \real{0.15}}\centering
4.2\strut
\end{minipage} & \begin{minipage}[t]{(\columnwidth - 5\tabcolsep) * \real{0.15}}\centering
-2.4\strut
\end{minipage} & \begin{minipage}[t]{(\columnwidth - 5\tabcolsep) * \real{0.20}}\centering
23\strut
\end{minipage}\tabularnewline
\begin{minipage}[t]{(\columnwidth - 5\tabcolsep) * \real{0.35}}\centering
Sutton.Bonnington:1959:02\strut
\end{minipage} & \begin{minipage}[t]{(\columnwidth - 5\tabcolsep) * \real{0.09}}\centering
1959\strut
\end{minipage} & \begin{minipage}[t]{(\columnwidth - 5\tabcolsep) * \real{0.06}}\centering
02\strut
\end{minipage} & \begin{minipage}[t]{(\columnwidth - 5\tabcolsep) * \real{0.15}}\centering
7.0\strut
\end{minipage} & \begin{minipage}[t]{(\columnwidth - 5\tabcolsep) * \real{0.15}}\centering
0.9\strut
\end{minipage} & \begin{minipage}[t]{(\columnwidth - 5\tabcolsep) * \real{0.20}}\centering
11\strut
\end{minipage}\tabularnewline
\begin{minipage}[t]{(\columnwidth - 5\tabcolsep) * \real{0.35}}\centering
Sutton.Bonnington:1959:03\strut
\end{minipage} & \begin{minipage}[t]{(\columnwidth - 5\tabcolsep) * \real{0.09}}\centering
1959\strut
\end{minipage} & \begin{minipage}[t]{(\columnwidth - 5\tabcolsep) * \real{0.06}}\centering
03\strut
\end{minipage} & \begin{minipage}[t]{(\columnwidth - 5\tabcolsep) * \real{0.15}}\centering
10.7\strut
\end{minipage} & \begin{minipage}[t]{(\columnwidth - 5\tabcolsep) * \real{0.15}}\centering
3.2\strut
\end{minipage} & \begin{minipage}[t]{(\columnwidth - 5\tabcolsep) * \real{0.20}}\centering
2\strut
\end{minipage}\tabularnewline
\begin{minipage}[t]{(\columnwidth - 5\tabcolsep) * \real{0.35}}\centering
Sutton.Bonnington:1959:04\strut
\end{minipage} & \begin{minipage}[t]{(\columnwidth - 5\tabcolsep) * \real{0.09}}\centering
1959\strut
\end{minipage} & \begin{minipage}[t]{(\columnwidth - 5\tabcolsep) * \real{0.06}}\centering
04\strut
\end{minipage} & \begin{minipage}[t]{(\columnwidth - 5\tabcolsep) * \real{0.15}}\centering
13.8\strut
\end{minipage} & \begin{minipage}[t]{(\columnwidth - 5\tabcolsep) * \real{0.15}}\centering
5.1\strut
\end{minipage} & \begin{minipage}[t]{(\columnwidth - 5\tabcolsep) * \real{0.20}}\centering
0\strut
\end{minipage}\tabularnewline
\begin{minipage}[t]{(\columnwidth - 5\tabcolsep) * \real{0.35}}\centering
Sutton.Bonnington:1959:05\strut
\end{minipage} & \begin{minipage}[t]{(\columnwidth - 5\tabcolsep) * \real{0.09}}\centering
1959\strut
\end{minipage} & \begin{minipage}[t]{(\columnwidth - 5\tabcolsep) * \real{0.06}}\centering
05\strut
\end{minipage} & \begin{minipage}[t]{(\columnwidth - 5\tabcolsep) * \real{0.15}}\centering
17.6\strut
\end{minipage} & \begin{minipage}[t]{(\columnwidth - 5\tabcolsep) * \real{0.15}}\centering
6.1\strut
\end{minipage} & \begin{minipage}[t]{(\columnwidth - 5\tabcolsep) * \real{0.20}}\centering
3\strut
\end{minipage}\tabularnewline
\begin{minipage}[t]{(\columnwidth - 5\tabcolsep) * \real{0.35}}\centering
Sutton.Bonnington:1959:06\strut
\end{minipage} & \begin{minipage}[t]{(\columnwidth - 5\tabcolsep) * \real{0.09}}\centering
1959\strut
\end{minipage} & \begin{minipage}[t]{(\columnwidth - 5\tabcolsep) * \real{0.06}}\centering
06\strut
\end{minipage} & \begin{minipage}[t]{(\columnwidth - 5\tabcolsep) * \real{0.15}}\centering
20.6\strut
\end{minipage} & \begin{minipage}[t]{(\columnwidth - 5\tabcolsep) * \real{0.15}}\centering
9.7\strut
\end{minipage} & \begin{minipage}[t]{(\columnwidth - 5\tabcolsep) * \real{0.20}}\centering
0\strut
\end{minipage}\tabularnewline
\bottomrule
\end{longtable}

\begin{longtable}[]{@{}cccccc@{}}
\caption{Table continues below}\tabularnewline
\toprule
\begin{minipage}[b]{(\columnwidth - 5\tabcolsep) * \real{0.15}}\centering
sun.hours\strut
\end{minipage} & \begin{minipage}[b]{(\columnwidth - 5\tabcolsep) * \real{0.16}}\centering
YYYYMMDD\strut
\end{minipage} & \begin{minipage}[b]{(\columnwidth - 5\tabcolsep) * \real{0.25}}\centering
place\strut
\end{minipage} & \begin{minipage}[b]{(\columnwidth - 5\tabcolsep) * \real{0.15}}\centering
lattitude\strut
\end{minipage} & \begin{minipage}[b]{(\columnwidth - 5\tabcolsep) * \real{0.14}}\centering
logitude\strut
\end{minipage} & \begin{minipage}[b]{(\columnwidth - 5\tabcolsep) * \real{0.14}}\centering
easting\strut
\end{minipage}\tabularnewline
\midrule
\endfirsthead
\toprule
\begin{minipage}[b]{(\columnwidth - 5\tabcolsep) * \real{0.15}}\centering
sun.hours\strut
\end{minipage} & \begin{minipage}[b]{(\columnwidth - 5\tabcolsep) * \real{0.16}}\centering
YYYYMMDD\strut
\end{minipage} & \begin{minipage}[b]{(\columnwidth - 5\tabcolsep) * \real{0.25}}\centering
place\strut
\end{minipage} & \begin{minipage}[b]{(\columnwidth - 5\tabcolsep) * \real{0.15}}\centering
lattitude\strut
\end{minipage} & \begin{minipage}[b]{(\columnwidth - 5\tabcolsep) * \real{0.14}}\centering
logitude\strut
\end{minipage} & \begin{minipage}[b]{(\columnwidth - 5\tabcolsep) * \real{0.14}}\centering
easting\strut
\end{minipage}\tabularnewline
\midrule
\endhead
\begin{minipage}[t]{(\columnwidth - 5\tabcolsep) * \real{0.15}}\centering
78.8\strut
\end{minipage} & \begin{minipage}[t]{(\columnwidth - 5\tabcolsep) * \real{0.16}}\centering
1959-01-31\strut
\end{minipage} & \begin{minipage}[t]{(\columnwidth - 5\tabcolsep) * \real{0.25}}\centering
Sutton Bonnington\strut
\end{minipage} & \begin{minipage}[t]{(\columnwidth - 5\tabcolsep) * \real{0.15}}\centering
52.833\strut
\end{minipage} & \begin{minipage}[t]{(\columnwidth - 5\tabcolsep) * \real{0.14}}\centering
-1.25\strut
\end{minipage} & \begin{minipage}[t]{(\columnwidth - 5\tabcolsep) * \real{0.14}}\centering
450700\strut
\end{minipage}\tabularnewline
\begin{minipage}[t]{(\columnwidth - 5\tabcolsep) * \real{0.15}}\centering
54.0\strut
\end{minipage} & \begin{minipage}[t]{(\columnwidth - 5\tabcolsep) * \real{0.16}}\centering
1959-02-28\strut
\end{minipage} & \begin{minipage}[t]{(\columnwidth - 5\tabcolsep) * \real{0.25}}\centering
Sutton Bonnington\strut
\end{minipage} & \begin{minipage}[t]{(\columnwidth - 5\tabcolsep) * \real{0.15}}\centering
52.833\strut
\end{minipage} & \begin{minipage}[t]{(\columnwidth - 5\tabcolsep) * \real{0.14}}\centering
-1.25\strut
\end{minipage} & \begin{minipage}[t]{(\columnwidth - 5\tabcolsep) * \real{0.14}}\centering
450700\strut
\end{minipage}\tabularnewline
\begin{minipage}[t]{(\columnwidth - 5\tabcolsep) * \real{0.15}}\centering
80.7\strut
\end{minipage} & \begin{minipage}[t]{(\columnwidth - 5\tabcolsep) * \real{0.16}}\centering
1959-03-31\strut
\end{minipage} & \begin{minipage}[t]{(\columnwidth - 5\tabcolsep) * \real{0.25}}\centering
Sutton Bonnington\strut
\end{minipage} & \begin{minipage}[t]{(\columnwidth - 5\tabcolsep) * \real{0.15}}\centering
52.833\strut
\end{minipage} & \begin{minipage}[t]{(\columnwidth - 5\tabcolsep) * \real{0.14}}\centering
-1.25\strut
\end{minipage} & \begin{minipage}[t]{(\columnwidth - 5\tabcolsep) * \real{0.14}}\centering
450700\strut
\end{minipage}\tabularnewline
\begin{minipage}[t]{(\columnwidth - 5\tabcolsep) * \real{0.15}}\centering
148.1\strut
\end{minipage} & \begin{minipage}[t]{(\columnwidth - 5\tabcolsep) * \real{0.16}}\centering
1959-04-30\strut
\end{minipage} & \begin{minipage}[t]{(\columnwidth - 5\tabcolsep) * \real{0.25}}\centering
Sutton Bonnington\strut
\end{minipage} & \begin{minipage}[t]{(\columnwidth - 5\tabcolsep) * \real{0.15}}\centering
52.833\strut
\end{minipage} & \begin{minipage}[t]{(\columnwidth - 5\tabcolsep) * \real{0.14}}\centering
-1.25\strut
\end{minipage} & \begin{minipage}[t]{(\columnwidth - 5\tabcolsep) * \real{0.14}}\centering
450700\strut
\end{minipage}\tabularnewline
\begin{minipage}[t]{(\columnwidth - 5\tabcolsep) * \real{0.15}}\centering
206.7\strut
\end{minipage} & \begin{minipage}[t]{(\columnwidth - 5\tabcolsep) * \real{0.16}}\centering
1959-05-31\strut
\end{minipage} & \begin{minipage}[t]{(\columnwidth - 5\tabcolsep) * \real{0.25}}\centering
Sutton Bonnington\strut
\end{minipage} & \begin{minipage}[t]{(\columnwidth - 5\tabcolsep) * \real{0.15}}\centering
52.833\strut
\end{minipage} & \begin{minipage}[t]{(\columnwidth - 5\tabcolsep) * \real{0.14}}\centering
-1.25\strut
\end{minipage} & \begin{minipage}[t]{(\columnwidth - 5\tabcolsep) * \real{0.14}}\centering
450700\strut
\end{minipage}\tabularnewline
\begin{minipage}[t]{(\columnwidth - 5\tabcolsep) * \real{0.15}}\centering
246.6\strut
\end{minipage} & \begin{minipage}[t]{(\columnwidth - 5\tabcolsep) * \real{0.16}}\centering
1959-06-30\strut
\end{minipage} & \begin{minipage}[t]{(\columnwidth - 5\tabcolsep) * \real{0.25}}\centering
Sutton Bonnington\strut
\end{minipage} & \begin{minipage}[t]{(\columnwidth - 5\tabcolsep) * \real{0.15}}\centering
52.833\strut
\end{minipage} & \begin{minipage}[t]{(\columnwidth - 5\tabcolsep) * \real{0.14}}\centering
-1.25\strut
\end{minipage} & \begin{minipage}[t]{(\columnwidth - 5\tabcolsep) * \real{0.14}}\centering
450700\strut
\end{minipage}\tabularnewline
\bottomrule
\end{longtable}

\begin{longtable}[]{@{}ccc@{}}
\toprule
\begin{minipage}[b]{(\columnwidth - 2\tabcolsep) * \real{0.15}}\centering
northing\strut
\end{minipage} & \begin{minipage}[b]{(\columnwidth - 2\tabcolsep) * \real{0.28}}\centering
height.amsl.metre\strut
\end{minipage} & \begin{minipage}[b]{(\columnwidth - 2\tabcolsep) * \real{0.14}}\centering
rain.mm\strut
\end{minipage}\tabularnewline
\midrule
\endhead
\begin{minipage}[t]{(\columnwidth - 2\tabcolsep) * \real{0.15}}\centering
325900\strut
\end{minipage} & \begin{minipage}[t]{(\columnwidth - 2\tabcolsep) * \real{0.28}}\centering
48\strut
\end{minipage} & \begin{minipage}[t]{(\columnwidth - 2\tabcolsep) * \real{0.14}}\centering
NA\strut
\end{minipage}\tabularnewline
\begin{minipage}[t]{(\columnwidth - 2\tabcolsep) * \real{0.15}}\centering
325900\strut
\end{minipage} & \begin{minipage}[t]{(\columnwidth - 2\tabcolsep) * \real{0.28}}\centering
48\strut
\end{minipage} & \begin{minipage}[t]{(\columnwidth - 2\tabcolsep) * \real{0.14}}\centering
NA\strut
\end{minipage}\tabularnewline
\begin{minipage}[t]{(\columnwidth - 2\tabcolsep) * \real{0.15}}\centering
325900\strut
\end{minipage} & \begin{minipage}[t]{(\columnwidth - 2\tabcolsep) * \real{0.28}}\centering
48\strut
\end{minipage} & \begin{minipage}[t]{(\columnwidth - 2\tabcolsep) * \real{0.14}}\centering
NA\strut
\end{minipage}\tabularnewline
\begin{minipage}[t]{(\columnwidth - 2\tabcolsep) * \real{0.15}}\centering
325900\strut
\end{minipage} & \begin{minipage}[t]{(\columnwidth - 2\tabcolsep) * \real{0.28}}\centering
48\strut
\end{minipage} & \begin{minipage}[t]{(\columnwidth - 2\tabcolsep) * \real{0.14}}\centering
NA\strut
\end{minipage}\tabularnewline
\begin{minipage}[t]{(\columnwidth - 2\tabcolsep) * \real{0.15}}\centering
325900\strut
\end{minipage} & \begin{minipage}[t]{(\columnwidth - 2\tabcolsep) * \real{0.28}}\centering
48\strut
\end{minipage} & \begin{minipage}[t]{(\columnwidth - 2\tabcolsep) * \real{0.14}}\centering
NA\strut
\end{minipage}\tabularnewline
\begin{minipage}[t]{(\columnwidth - 2\tabcolsep) * \real{0.15}}\centering
325900\strut
\end{minipage} & \begin{minipage}[t]{(\columnwidth - 2\tabcolsep) * \real{0.28}}\centering
48\strut
\end{minipage} & \begin{minipage}[t]{(\columnwidth - 2\tabcolsep) * \real{0.14}}\centering
NA\strut
\end{minipage}\tabularnewline
\bottomrule
\end{longtable}

To make out strictly defined data to analyse the weather we select the columns of interest and make a long format with the date as the key column. We choose this format as it gives great flexibility with analysis and works well with a Grammar of Graphic (GoG) approach. This contrasts with the temptation to produce a wide format of data as one might use in a spreadsheet analysis. The versatility of GoG will become apparent as we proceed and see how all plots may be specified by changing the GoG verbs.

\begin{Shaded}
\begin{Highlighting}[]
\CommentTok{\# But we are interesting in plotting the data by date so}

\NormalTok{  VC55.weather.ss }\OtherTok{\textless{}{-}}\NormalTok{ VC55.weather.sl }\SpecialCharTok{\%\textgreater{}\%} 
    \FunctionTok{select}\NormalTok{(}\StringTok{"YYYYMMDD"}\NormalTok{, }
           \StringTok{"tmax.degC"}\NormalTok{,}
           \StringTok{"tmin.degC"}\NormalTok{,}
           \StringTok{"airfrost.days"}\NormalTok{, }
           \StringTok{"sun.hours"}\NormalTok{ ,}
           \StringTok{"rain.mm"}\NormalTok{)     }\SpecialCharTok{\%\textgreater{}\%}
            \FunctionTok{pivot\_longer}\NormalTok{(}\SpecialCharTok{!}\NormalTok{YYYYMMDD,}\CommentTok{\# Use datumEntity as the key}
                  \AttributeTok{names\_to =} \StringTok{"datumAttribute"}\NormalTok{, }\CommentTok{\# Save the name of the attribute}
                  \AttributeTok{values\_to =} \StringTok{"datumValue"}\NormalTok{) }\CommentTok{\# Save the value of the attribute}
\end{Highlighting}
\end{Shaded}

At this point we can specify data types as Date and numeric for analysis.

\begin{Shaded}
\begin{Highlighting}[]
\CommentTok{\# Set the data types}
\NormalTok{VC55.weather.ss}\SpecialCharTok{$}\NormalTok{YYYYMMDD }\OtherTok{\textless{}{-}}\NormalTok{  VC55.weather.ss}\SpecialCharTok{$}\NormalTok{YYYYMMDD }\SpecialCharTok{\%\textgreater{}\%}\NormalTok{ as.Date}
\NormalTok{VC55.weather.ss}\SpecialCharTok{$}\NormalTok{datumValue }\OtherTok{\textless{}{-}}\NormalTok{ VC55.weather.ss}\SpecialCharTok{$}\NormalTok{datumValue }\SpecialCharTok{\%\textgreater{}\%} \FunctionTok{as.numeric}\NormalTok{()}
\end{Highlighting}
\end{Shaded}

We can now save into \texttt{data-ss}.

\begin{Shaded}
\begin{Highlighting}[]
\FunctionTok{dir.create}\NormalTok{( }
\NormalTok{          fs}\SpecialCharTok{::}\FunctionTok{path}\NormalTok{( absolute.path,}
            \StringTok{"data{-}ss"}\NormalTok{ ),  }
            \AttributeTok{showWarnings =} \ConstantTok{FALSE}\NormalTok{,}
            \AttributeTok{recursive =} \ConstantTok{TRUE}\NormalTok{) }\CommentTok{\# Create the directory}

\CommentTok{\#And save.}
\NormalTok{path.to.save.ss}\OtherTok{\textless{}{-}}\NormalTok{ fs}\SpecialCharTok{::}\FunctionTok{path}\NormalTok{( absolute.path}
\NormalTok{                               ,}\StringTok{"data{-}ss"}\NormalTok{,}\StringTok{"VC55.weather.ss"}\NormalTok{, }\AttributeTok{ext =} \StringTok{"rds"}\NormalTok{)}
\FunctionTok{saveRDS}\NormalTok{(VC55.weather.ss, path.to.save.ss)}
\end{Highlighting}
\end{Shaded}

Once again the data may be loaded and the analysis start from this point. Rather than load as a named file, we demonstrate how multiple files in \texttt{data-ss} may be loaded and combined in a simple action. Since \texttt{data-ss} are strictly defined each file may be `stacked' to combine into a larger data-set. If this were really the case then it would make sense to rename the data to soemthing more appropriate. The named method of loading is included as comments for comparison.

\begin{Shaded}
\begin{Highlighting}[]
\CommentTok{\# Load data{-}ss}
\NormalTok{path.to.data.ss }\OtherTok{\textless{}{-}}\NormalTok{ fs}\SpecialCharTok{::}\FunctionTok{path}\NormalTok{(absolute.path,}\StringTok{"data{-}ss"}\NormalTok{)}

\NormalTok{VC55.weather.ss }\OtherTok{\textless{}{-}} \FunctionTok{list.files}\NormalTok{(}
\NormalTok{  path.to.data.ss,}
  \AttributeTok{pattern =} \StringTok{".rds"}\NormalTok{, }\CommentTok{\# Make a suitable filter. Use the dot for a wildcard.}
  \AttributeTok{full.names =} \ConstantTok{TRUE}\NormalTok{,}
  \AttributeTok{recursive =} \ConstantTok{TRUE}\NormalTok{)  }\SpecialCharTok{\%\textgreater{}\%}
\NormalTok{  purrr}\SpecialCharTok{::}\FunctionTok{map\_df}\NormalTok{(readRDS) }

\CommentTok{\# path.to.data \textless{}{-} fs::path( absolute.path}
\CommentTok{\#                                ,"data{-}ss","suttonboningtondata", ext = "rds")}
\CommentTok{\# \# Could change the name of the data if we so wished.}
\CommentTok{\# suttonboningtondata \textless{}{-}  readRDS(path.to.data)}
\end{Highlighting}
\end{Shaded}

At this point we can do a simple check plot which uncovers some issues. By faceting on \texttt{datumAttribute} we can produce a separate graph for each attribute. While is shows we have data, each graph has different units, so the scales do not male sense.

\begin{Shaded}
\begin{Highlighting}[]
\CommentTok{\# Check plot}

\NormalTok{data.to.plot }\OtherTok{\textless{}{-}}\NormalTok{ VC55.weather.ss}

\NormalTok{data.to.plot }\SpecialCharTok{\%\textgreater{}\%} \FunctionTok{drop\_na}\NormalTok{() }\SpecialCharTok{\%\textgreater{}\%}
\FunctionTok{ggplot}\NormalTok{(}\FunctionTok{aes}\NormalTok{(}\AttributeTok{colour =}\NormalTok{ datumAttribute, }\AttributeTok{x=}\NormalTok{ YYYYMMDD, }\AttributeTok{y =}\NormalTok{datumValue, }\AttributeTok{group =}\NormalTok{ datumAttribute) ) }\SpecialCharTok{+} 
 \CommentTok{\# geom\_point() +}
  \FunctionTok{geom\_smooth}\NormalTok{( }\AttributeTok{se =} \ConstantTok{FALSE}\NormalTok{, }\AttributeTok{method =} \StringTok{"loess"}\NormalTok{, }\AttributeTok{formula =} \StringTok{"y \textasciitilde{} x"}\NormalTok{) }\SpecialCharTok{+}
  \FunctionTok{facet\_wrap}\NormalTok{(}\SpecialCharTok{\textasciitilde{}}\NormalTok{ datumAttribute)}
\end{Highlighting}
\end{Shaded}

\begin{center}\includegraphics[width=0.7\linewidth]{Sutton_Bonnington_weather_files/figure-latex/unnamed-chunk-20-1} \end{center}

\begin{Shaded}
\begin{Highlighting}[]
 \CommentTok{\#   geom\_line() + facet\_wrap(\textasciitilde{} datumAttribute)}
\end{Highlighting}
\end{Shaded}

However, if we use z-scores instead then each plot is normalised against zero and the standard deviation. Rather than 5 separate plots, we can use a single graph and plot a smooth loess regression.

\begin{Shaded}
\begin{Highlighting}[]
\CommentTok{\# An improved summary}


\NormalTok{data.to.plot }\OtherTok{\textless{}{-}}\NormalTok{ VC55.weather.ss}
\NormalTok{ data.to.plot }\SpecialCharTok{\%\textgreater{}\%} \FunctionTok{drop\_na}\NormalTok{() }\SpecialCharTok{\%\textgreater{}\%}
  \FunctionTok{group\_by}\NormalTok{(datumAttribute) }\SpecialCharTok{\%\textgreater{}\%} 
  \FunctionTok{mutate}\NormalTok{( }\AttributeTok{value =}\NormalTok{ datumValue,}
          \AttributeTok{Z.score =}\NormalTok{ (datumValue }\SpecialCharTok{{-}}\FunctionTok{mean}\NormalTok{(datumValue))}\SpecialCharTok{/}\FunctionTok{sd}\NormalTok{(datumValue)}
\NormalTok{        ) }\SpecialCharTok{\%\textgreater{}\%}
  \FunctionTok{ggplot}\NormalTok{( }\FunctionTok{aes}\NormalTok{(}\AttributeTok{colour =}\NormalTok{ datumAttribute, }
              \AttributeTok{x=}\NormalTok{ YYYYMMDD, }\AttributeTok{y =}\NormalTok{Z.score, }
              \AttributeTok{group =}\NormalTok{ datumAttribute) ) }\SpecialCharTok{+} 
              \FunctionTok{geom\_smooth}\NormalTok{( }\AttributeTok{se =} \ConstantTok{FALSE}\NormalTok{, }\AttributeTok{method =} \StringTok{"loess"}\NormalTok{, }\AttributeTok{formula =} \StringTok{"y \textasciitilde{} x"}\NormalTok{) }\SpecialCharTok{+} 
 \CommentTok{\#   geom\_point(data =data.to.plot[abs(data.to.plot$Z.score) \textgreater{}= 1.5,]) +}
              \FunctionTok{scale\_colour\_viridis\_d}\NormalTok{() }
\end{Highlighting}
\end{Shaded}

\begin{center}\includegraphics[width=0.7\linewidth]{Sutton_Bonnington_weather_files/figure-latex/unnamed-chunk-21-1} \end{center}

This looks good but it is predicated upon the deviations being normally distributed about the mean. We can check this with a QQ plot.

\begin{Shaded}
\begin{Highlighting}[]
\CommentTok{\# An improved summary}
\NormalTok{my.dist }\OtherTok{\textless{}{-}} \StringTok{"norm"}

\NormalTok{data.to.plot }\OtherTok{\textless{}{-}}\NormalTok{ VC55.weather.ss}
\NormalTok{ data.to.plot }\SpecialCharTok{\%\textgreater{}\%} \FunctionTok{drop\_na}\NormalTok{() }\SpecialCharTok{\%\textgreater{}\%}
  \FunctionTok{group\_by}\NormalTok{(datumAttribute) }\SpecialCharTok{\%\textgreater{}\%} 
  \FunctionTok{mutate}\NormalTok{( }\AttributeTok{value =}\NormalTok{ datumValue,}
          \AttributeTok{Z.score =}\NormalTok{ (datumValue }\SpecialCharTok{{-}}\FunctionTok{mean}\NormalTok{(datumValue))}\SpecialCharTok{/}\FunctionTok{sd}\NormalTok{(datumValue)}
\NormalTok{        ) }\SpecialCharTok{\%\textgreater{}\%}
 \CommentTok{\#  ggplot( aes(colour = datumAttribute, }
 \CommentTok{\#              x= YYYYMMDD, y =Z.score, }
 \CommentTok{\#              group = datumAttribute) ) + }
 \CommentTok{\#              geom\_smooth( se = FALSE, method = "loess", formula = "y \textasciitilde{} x") + }
 \CommentTok{\# \#   geom\_point(data =data.to.plot[abs(data.to.plot$Z.score) \textgreater{}= 1.5,]) +}
 \CommentTok{\#              scale\_colour\_viridis\_d() }



\CommentTok{\#ggplot(distribution.data, aes(x=centrality)) + stat\_qq( sample = norm)}

\FunctionTok{ggplot}\NormalTok{(}\AttributeTok{mapping =} \FunctionTok{aes}\NormalTok{( }
                      \AttributeTok{group =}\NormalTok{ datumAttribute,}
                      \AttributeTok{sample =}\NormalTok{ Z.score)) }\SpecialCharTok{+}
 \CommentTok{\#   stat\_qq\_band(distribution = my.dist) +}
    \FunctionTok{stat\_qq\_line}\NormalTok{(}\AttributeTok{distribution =}\NormalTok{ my.dist) }\SpecialCharTok{+}
   \FunctionTok{stat\_qq\_point}\NormalTok{(}\AttributeTok{distribution =}\NormalTok{ my.dist) }\SpecialCharTok{+} 
  \FunctionTok{scale\_colour\_viridis\_d}\NormalTok{() }\SpecialCharTok{+}
  \FunctionTok{ggtitle}\NormalTok{(}\StringTok{"Z{-}score compared to normal distribution"}\NormalTok{) }\SpecialCharTok{+} \FunctionTok{facet\_wrap}\NormalTok{(}\SpecialCharTok{\textasciitilde{}}\NormalTok{ datumAttribute)}
\end{Highlighting}
\end{Shaded}

\begin{center}\includegraphics[width=0.7\linewidth]{Sutton_Bonnington_weather_files/figure-latex/unnamed-chunk-22-1} \end{center}

Our assumption or a normal distribution is reasonably valid, but not perfect. As we are looking at weather data, it might be nice to look at: Annual Total Rainfall, Annual Air Frost days, Annual Maximum Temperature, and Annual Minimum Temperature. Again GoG comes to the rescue and we can quickly produce the following plots.

\hypertarget{rainfall}{%
\subsection{Rainfall}\label{rainfall}}

\begin{Shaded}
\begin{Highlighting}[]
\CommentTok{\# Weather "tmax.degC"     "tmin.degC"     "airfrost.days" "sun.hours"     "rain.mm"}

\NormalTok{weather.attribute }\OtherTok{\textless{}{-}} \StringTok{"rain.mm"}

\NormalTok{path.to.ss}\OtherTok{\textless{}{-}}\NormalTok{ fs}\SpecialCharTok{::}\FunctionTok{path}\NormalTok{( absolute.path}
\NormalTok{                               ,}\StringTok{"data{-}ss"}\NormalTok{,}\StringTok{"VC55.weather.ss"}\NormalTok{, }\AttributeTok{ext =} \StringTok{"rds"}\NormalTok{)}
\NormalTok{VC55.weather.ss  }\OtherTok{\textless{}{-}}   \FunctionTok{readRDS}\NormalTok{(path.to.ss)}

\NormalTok{data.to.plot }\OtherTok{\textless{}{-}}\NormalTok{ VC55.weather.ss[VC55.weather.ss}\SpecialCharTok{$}\NormalTok{datumAttribute }\SpecialCharTok{==}\NormalTok{ weather.attribute, ]}

\NormalTok{data.to.plot}\SpecialCharTok{$}\NormalTok{YYYY }\OtherTok{\textless{}{-}} \FunctionTok{format}\NormalTok{(data.to.plot}\SpecialCharTok{$}\NormalTok{YYYYMMDD, }\AttributeTok{format =} \StringTok{"\%Y"}\NormalTok{) }\SpecialCharTok{\%\textgreater{}\%} \FunctionTok{as.numeric}\NormalTok{()}

\NormalTok{data.to.plot.annual }\OtherTok{\textless{}{-}}\NormalTok{ data.to.plot }\SpecialCharTok{\%\textgreater{}\%} \FunctionTok{drop\_na}\NormalTok{() }\SpecialCharTok{\%\textgreater{}\%}
    \FunctionTok{group\_by}\NormalTok{(YYYY) }\SpecialCharTok{\%\textgreater{}\%}
  \FunctionTok{mutate}\NormalTok{( }
          \AttributeTok{annual =} \FunctionTok{sum}\NormalTok{(datumValue)}
\NormalTok{        ) }
  
\FunctionTok{ggplot}\NormalTok{(}\AttributeTok{data =}\NormalTok{data.to.plot.annual, }\FunctionTok{aes}\NormalTok{( }\AttributeTok{x=}\NormalTok{ YYYY, }\AttributeTok{y =}\NormalTok{annual) ) }\SpecialCharTok{+} 
  \FunctionTok{geom\_point}\NormalTok{() }\SpecialCharTok{+}
\CommentTok{\#   geom\_smooth( se = FALSE, method = "loess", formula = "y \textasciitilde{} x") +}
  \FunctionTok{geom\_line}\NormalTok{()}\SpecialCharTok{+} \FunctionTok{ggtitle}\NormalTok{(weather.attribute) }\SpecialCharTok{+} \FunctionTok{ylab}\NormalTok{(weather.attribute)}
\end{Highlighting}
\end{Shaded}

\begin{center}\includegraphics[width=0.7\linewidth]{Sutton_Bonnington_weather_files/figure-latex/unnamed-chunk-23-1} \end{center}

\hypertarget{airfrost}{%
\subsection{Airfrost}\label{airfrost}}

\begin{Shaded}
\begin{Highlighting}[]
\CommentTok{\# Weather "tmax.degC"     "tmin.degC"     "airfrost.days" "sun.hours"     "rain.mm"  }

\NormalTok{weather.attribute }\OtherTok{\textless{}{-}} \StringTok{"airfrost.days"}

\NormalTok{path.to.ss}\OtherTok{\textless{}{-}}\NormalTok{ fs}\SpecialCharTok{::}\FunctionTok{path}\NormalTok{( absolute.path}
\NormalTok{                               ,}\StringTok{"data{-}ss"}\NormalTok{,}\StringTok{"VC55.weather.ss"}\NormalTok{, }\AttributeTok{ext =} \StringTok{"rds"}\NormalTok{)}
\NormalTok{VC55.weather.ss  }\OtherTok{\textless{}{-}}   \FunctionTok{readRDS}\NormalTok{(path.to.ss)}

\NormalTok{data.to.plot }\OtherTok{\textless{}{-}}\NormalTok{ VC55.weather.ss[VC55.weather.ss}\SpecialCharTok{$}\NormalTok{datumAttribute }\SpecialCharTok{==}\NormalTok{ weather.attribute, ]}

\NormalTok{data.to.plot}\SpecialCharTok{$}\NormalTok{YYYY }\OtherTok{\textless{}{-}} \FunctionTok{format}\NormalTok{(data.to.plot}\SpecialCharTok{$}\NormalTok{YYYYMMDD, }\AttributeTok{format =} \StringTok{"\%Y"}\NormalTok{) }\SpecialCharTok{\%\textgreater{}\%} \FunctionTok{as.numeric}\NormalTok{()}

\NormalTok{data.to.plot.annual }\OtherTok{\textless{}{-}}\NormalTok{ data.to.plot }\SpecialCharTok{\%\textgreater{}\%} \FunctionTok{drop\_na}\NormalTok{() }\SpecialCharTok{\%\textgreater{}\%}
    \FunctionTok{group\_by}\NormalTok{(YYYY) }\SpecialCharTok{\%\textgreater{}\%}
  \FunctionTok{mutate}\NormalTok{( }
          \AttributeTok{annual =} \FunctionTok{sum}\NormalTok{(datumValue)}
\NormalTok{        ) }
  
\FunctionTok{ggplot}\NormalTok{(}\AttributeTok{data =}\NormalTok{data.to.plot.annual, }\FunctionTok{aes}\NormalTok{( }\AttributeTok{x=}\NormalTok{ YYYY, }\AttributeTok{y =}\NormalTok{annual) ) }\SpecialCharTok{+} 
  \FunctionTok{geom\_point}\NormalTok{() }\SpecialCharTok{+}
\CommentTok{\#   geom\_smooth( se = FALSE, method = "loess", formula = "y \textasciitilde{} x") +}
  \FunctionTok{geom\_line}\NormalTok{()}\SpecialCharTok{+} \FunctionTok{ggtitle}\NormalTok{(weather.attribute) }\SpecialCharTok{+} \FunctionTok{ylab}\NormalTok{(weather.attribute)}
\end{Highlighting}
\end{Shaded}

\begin{center}\includegraphics[width=0.7\linewidth]{Sutton_Bonnington_weather_files/figure-latex/unnamed-chunk-24-1} \end{center}

\hypertarget{maximum-temperature}{%
\subsection{Maximum Temperature}\label{maximum-temperature}}

\begin{Shaded}
\begin{Highlighting}[]
\CommentTok{\# Weather "tmax.degC"     "tmin.degC"     "airfrost.days" "sun.hours"     "rain.mm"  }

\NormalTok{weather.attribute }\OtherTok{\textless{}{-}} \StringTok{"tmax.degC"}

\NormalTok{path.to.ss}\OtherTok{\textless{}{-}}\NormalTok{ fs}\SpecialCharTok{::}\FunctionTok{path}\NormalTok{( absolute.path}
\NormalTok{                               ,}\StringTok{"data{-}ss"}\NormalTok{,}\StringTok{"VC55.weather.ss"}\NormalTok{, }\AttributeTok{ext =} \StringTok{"rds"}\NormalTok{)}
\NormalTok{VC55.weather.ss  }\OtherTok{\textless{}{-}}   \FunctionTok{readRDS}\NormalTok{(path.to.ss)}

\NormalTok{data.to.plot }\OtherTok{\textless{}{-}}\NormalTok{ VC55.weather.ss[VC55.weather.ss}\SpecialCharTok{$}\NormalTok{datumAttribute }\SpecialCharTok{==}\NormalTok{ weather.attribute, ]}

\NormalTok{data.to.plot}\SpecialCharTok{$}\NormalTok{YYYY }\OtherTok{\textless{}{-}} \FunctionTok{format}\NormalTok{(data.to.plot}\SpecialCharTok{$}\NormalTok{YYYYMMDD, }\AttributeTok{format =} \StringTok{"\%Y"}\NormalTok{) }\SpecialCharTok{\%\textgreater{}\%} \FunctionTok{as.numeric}\NormalTok{()}

\NormalTok{data.to.plot.annual }\OtherTok{\textless{}{-}}\NormalTok{ data.to.plot }\SpecialCharTok{\%\textgreater{}\%} \FunctionTok{drop\_na}\NormalTok{() }\SpecialCharTok{\%\textgreater{}\%}
    \FunctionTok{group\_by}\NormalTok{(YYYY) }\SpecialCharTok{\%\textgreater{}\%}
  \FunctionTok{mutate}\NormalTok{( }
          \AttributeTok{annual =} \FunctionTok{max}\NormalTok{(datumValue)}
\NormalTok{        ) }
  
\FunctionTok{ggplot}\NormalTok{(}\AttributeTok{data =}\NormalTok{data.to.plot.annual, }\FunctionTok{aes}\NormalTok{( }\AttributeTok{x=}\NormalTok{ YYYY, }\AttributeTok{y =}\NormalTok{annual) ) }\SpecialCharTok{+} 
  \FunctionTok{geom\_point}\NormalTok{() }\SpecialCharTok{+}
\CommentTok{\#   geom\_smooth( se = FALSE, method = "loess", formula = "y \textasciitilde{} x") +}
  \FunctionTok{geom\_line}\NormalTok{()}\SpecialCharTok{+} \FunctionTok{ggtitle}\NormalTok{(weather.attribute) }\SpecialCharTok{+} \FunctionTok{ylab}\NormalTok{(weather.attribute)}
\end{Highlighting}
\end{Shaded}

\begin{center}\includegraphics[width=0.7\linewidth]{Sutton_Bonnington_weather_files/figure-latex/unnamed-chunk-25-1} \end{center}

\hypertarget{minimum-temperature}{%
\subsection{Minimum Temperature}\label{minimum-temperature}}

\begin{Shaded}
\begin{Highlighting}[]
\CommentTok{\# Weather "tmax.degC"     "tmin.degC"     "airfrost.days" "sun.hours"     "rain.mm"  }

\NormalTok{weather.attribute }\OtherTok{\textless{}{-}} \StringTok{"tmin.degC"}

\NormalTok{path.to.ss}\OtherTok{\textless{}{-}}\NormalTok{ fs}\SpecialCharTok{::}\FunctionTok{path}\NormalTok{( absolute.path}
\NormalTok{                               ,}\StringTok{"data{-}ss"}\NormalTok{,}\StringTok{"VC55.weather.ss"}\NormalTok{, }\AttributeTok{ext =} \StringTok{"rds"}\NormalTok{)}
\NormalTok{VC55.weather.ss  }\OtherTok{\textless{}{-}}   \FunctionTok{readRDS}\NormalTok{(path.to.ss)}

\NormalTok{data.to.plot }\OtherTok{\textless{}{-}}\NormalTok{ VC55.weather.ss[VC55.weather.ss}\SpecialCharTok{$}\NormalTok{datumAttribute }\SpecialCharTok{==}\NormalTok{ weather.attribute, ]}

\NormalTok{data.to.plot}\SpecialCharTok{$}\NormalTok{YYYY }\OtherTok{\textless{}{-}} \FunctionTok{format}\NormalTok{(data.to.plot}\SpecialCharTok{$}\NormalTok{YYYYMMDD, }\AttributeTok{format =} \StringTok{"\%Y"}\NormalTok{) }\SpecialCharTok{\%\textgreater{}\%} \FunctionTok{as.numeric}\NormalTok{()}

\NormalTok{data.to.plot.annual }\OtherTok{\textless{}{-}}\NormalTok{ data.to.plot }\SpecialCharTok{\%\textgreater{}\%} \FunctionTok{drop\_na}\NormalTok{() }\SpecialCharTok{\%\textgreater{}\%}
    \FunctionTok{group\_by}\NormalTok{(YYYY) }\SpecialCharTok{\%\textgreater{}\%}
  \FunctionTok{mutate}\NormalTok{( }
          \AttributeTok{annual =} \FunctionTok{min}\NormalTok{(datumValue)}
\NormalTok{        ) }
  
\FunctionTok{ggplot}\NormalTok{(}\AttributeTok{data =}\NormalTok{data.to.plot.annual, }\FunctionTok{aes}\NormalTok{( }\AttributeTok{x=}\NormalTok{ YYYY, }\AttributeTok{y =}\NormalTok{annual) ) }\SpecialCharTok{+} 
  \FunctionTok{geom\_point}\NormalTok{() }\SpecialCharTok{+}
\CommentTok{\#   geom\_smooth( se = FALSE, method = "loess", formula = "y \textasciitilde{} x") +}
  \FunctionTok{geom\_line}\NormalTok{()}\SpecialCharTok{+} \FunctionTok{ggtitle}\NormalTok{(weather.attribute) }\SpecialCharTok{+} \FunctionTok{ylab}\NormalTok{(weather.attribute)}
\end{Highlighting}
\end{Shaded}

\begin{center}\includegraphics[width=0.7\linewidth]{Sutton_Bonnington_weather_files/figure-latex/unnamed-chunk-26-1} \end{center}

\hypertarget{conclusion}{%
\subsection{Conclusion}\label{conclusion}}

This vignette demonstrates the versatility of using data state in conjunction with GoG.

\hypertarget{references}{%
\section*{References}\label{references}}
\addcontentsline{toc}{section}{References}

\hypertarget{refs}{}
\begin{CSLReferences}{0}{0}
\end{CSLReferences}

\let\cleardoublepage\clearpage

\bibliographystyle{unsrt}
\bibliography{references.bib}


\end{document}
