% Options for packages loaded elsewhere
\PassOptionsToPackage{unicode}{hyperref}
\PassOptionsToPackage{hyphens}{url}
%
\documentclass[
  ignorenonframetext,
]{beamer}
\usepackage{pgfpages}
\setbeamertemplate{caption}[numbered]
\setbeamertemplate{caption label separator}{: }
\setbeamercolor{caption name}{fg=normal text.fg}
\beamertemplatenavigationsymbolsempty
% Prevent slide breaks in the middle of a paragraph
\widowpenalties 1 10000
\raggedbottom
\setbeamertemplate{part page}{
  \centering
  \begin{beamercolorbox}[sep=16pt,center]{part title}
    \usebeamerfont{part title}\insertpart\par
  \end{beamercolorbox}
}
\setbeamertemplate{section page}{
  \centering
  \begin{beamercolorbox}[sep=12pt,center]{part title}
    \usebeamerfont{section title}\insertsection\par
  \end{beamercolorbox}
}
\setbeamertemplate{subsection page}{
  \centering
  \begin{beamercolorbox}[sep=8pt,center]{part title}
    \usebeamerfont{subsection title}\insertsubsection\par
  \end{beamercolorbox}
}
\AtBeginPart{
  \frame{\partpage}
}
\AtBeginSection{
  \ifbibliography
  \else
    \frame{\sectionpage}
  \fi
}
\AtBeginSubsection{
  \frame{\subsectionpage}
}
\usepackage{amsmath,amssymb}
\usepackage{lmodern}
\usepackage{iftex}
\ifPDFTeX
  \usepackage[T1]{fontenc}
  \usepackage[utf8]{inputenc}
  \usepackage{textcomp} % provide euro and other symbols
\else % if luatex or xetex
  \usepackage{unicode-math}
  \defaultfontfeatures{Scale=MatchLowercase}
  \defaultfontfeatures[\rmfamily]{Ligatures=TeX,Scale=1}
\fi
\usetheme[]{AnnArbor}
\usecolortheme{dolphin}
\usefonttheme{structurebold}
% Use upquote if available, for straight quotes in verbatim environments
\IfFileExists{upquote.sty}{\usepackage{upquote}}{}
\IfFileExists{microtype.sty}{% use microtype if available
  \usepackage[]{microtype}
  \UseMicrotypeSet[protrusion]{basicmath} % disable protrusion for tt fonts
}{}
\makeatletter
\@ifundefined{KOMAClassName}{% if non-KOMA class
  \IfFileExists{parskip.sty}{%
    \usepackage{parskip}
  }{% else
    \setlength{\parindent}{0pt}
    \setlength{\parskip}{6pt plus 2pt minus 1pt}}
}{% if KOMA class
  \KOMAoptions{parskip=half}}
\makeatother
\usepackage{xcolor}
\IfFileExists{xurl.sty}{\usepackage{xurl}}{} % add URL line breaks if available
\IfFileExists{bookmark.sty}{\usepackage{bookmark}}{\usepackage{hyperref}}
\hypersetup{
  pdftitle={VC55 Weather Case Study},
  pdfauthor={Paul J. Palmer},
  hidelinks,
  pdfcreator={LaTeX via pandoc}}
\urlstyle{same} % disable monospaced font for URLs
\newif\ifbibliography
\usepackage{color}
\usepackage{fancyvrb}
\newcommand{\VerbBar}{|}
\newcommand{\VERB}{\Verb[commandchars=\\\{\}]}
\DefineVerbatimEnvironment{Highlighting}{Verbatim}{commandchars=\\\{\}}
% Add ',fontsize=\small' for more characters per line
\usepackage{framed}
\definecolor{shadecolor}{RGB}{248,248,248}
\newenvironment{Shaded}{\begin{snugshade}}{\end{snugshade}}
\newcommand{\AlertTok}[1]{\textcolor[rgb]{0.94,0.16,0.16}{#1}}
\newcommand{\AnnotationTok}[1]{\textcolor[rgb]{0.56,0.35,0.01}{\textbf{\textit{#1}}}}
\newcommand{\AttributeTok}[1]{\textcolor[rgb]{0.77,0.63,0.00}{#1}}
\newcommand{\BaseNTok}[1]{\textcolor[rgb]{0.00,0.00,0.81}{#1}}
\newcommand{\BuiltInTok}[1]{#1}
\newcommand{\CharTok}[1]{\textcolor[rgb]{0.31,0.60,0.02}{#1}}
\newcommand{\CommentTok}[1]{\textcolor[rgb]{0.56,0.35,0.01}{\textit{#1}}}
\newcommand{\CommentVarTok}[1]{\textcolor[rgb]{0.56,0.35,0.01}{\textbf{\textit{#1}}}}
\newcommand{\ConstantTok}[1]{\textcolor[rgb]{0.00,0.00,0.00}{#1}}
\newcommand{\ControlFlowTok}[1]{\textcolor[rgb]{0.13,0.29,0.53}{\textbf{#1}}}
\newcommand{\DataTypeTok}[1]{\textcolor[rgb]{0.13,0.29,0.53}{#1}}
\newcommand{\DecValTok}[1]{\textcolor[rgb]{0.00,0.00,0.81}{#1}}
\newcommand{\DocumentationTok}[1]{\textcolor[rgb]{0.56,0.35,0.01}{\textbf{\textit{#1}}}}
\newcommand{\ErrorTok}[1]{\textcolor[rgb]{0.64,0.00,0.00}{\textbf{#1}}}
\newcommand{\ExtensionTok}[1]{#1}
\newcommand{\FloatTok}[1]{\textcolor[rgb]{0.00,0.00,0.81}{#1}}
\newcommand{\FunctionTok}[1]{\textcolor[rgb]{0.00,0.00,0.00}{#1}}
\newcommand{\ImportTok}[1]{#1}
\newcommand{\InformationTok}[1]{\textcolor[rgb]{0.56,0.35,0.01}{\textbf{\textit{#1}}}}
\newcommand{\KeywordTok}[1]{\textcolor[rgb]{0.13,0.29,0.53}{\textbf{#1}}}
\newcommand{\NormalTok}[1]{#1}
\newcommand{\OperatorTok}[1]{\textcolor[rgb]{0.81,0.36,0.00}{\textbf{#1}}}
\newcommand{\OtherTok}[1]{\textcolor[rgb]{0.56,0.35,0.01}{#1}}
\newcommand{\PreprocessorTok}[1]{\textcolor[rgb]{0.56,0.35,0.01}{\textit{#1}}}
\newcommand{\RegionMarkerTok}[1]{#1}
\newcommand{\SpecialCharTok}[1]{\textcolor[rgb]{0.00,0.00,0.00}{#1}}
\newcommand{\SpecialStringTok}[1]{\textcolor[rgb]{0.31,0.60,0.02}{#1}}
\newcommand{\StringTok}[1]{\textcolor[rgb]{0.31,0.60,0.02}{#1}}
\newcommand{\VariableTok}[1]{\textcolor[rgb]{0.00,0.00,0.00}{#1}}
\newcommand{\VerbatimStringTok}[1]{\textcolor[rgb]{0.31,0.60,0.02}{#1}}
\newcommand{\WarningTok}[1]{\textcolor[rgb]{0.56,0.35,0.01}{\textbf{\textit{#1}}}}
\usepackage{longtable,booktabs,array}
\usepackage{calc} % for calculating minipage widths
\usepackage{caption}
% Make caption package work with longtable
\makeatletter
\def\fnum@table{\tablename~\thetable}
\makeatother
\setlength{\emergencystretch}{3em} % prevent overfull lines
\providecommand{\tightlist}{%
  \setlength{\itemsep}{0pt}\setlength{\parskip}{0pt}}
\setcounter{secnumdepth}{-\maxdimen} % remove section numbering
\title[ VC55 Weather Case Study]{VC55 Long Term Weather Data From Sutton Bonnington} % The short title appears at the bottom of every slide, the full title is only on the title page
\subtitle{A case study using data state to analyse inconvenient data}

%\author{Dr. Paul J. Palmer}

\institute[LU] % Your institution as it will appear on the bottom of every slide, may be shorthand to save space
{
    Visiting Fellow \\
    Loughborough University \\
    \copyright~P.J.~Palmer~2022
    
    
    
    % Your institution for the title page
}
\author{}
\date{\vspace{-2.5em}\today} % Date, can be changed to a custom date

  %title: ""
 % author: Dr. Paul J. Palmer
  %date:  "`r format(Sys.time(), '%d %B, %Y')`"
\ifLuaTeX
  \usepackage{selnolig}  % disable illegal ligatures
\fi

\title{VC55 Weather Case Study}
\author{Paul J. Palmer}
\date{}

\begin{document}
\frame{\titlepage}

\begin{frame}{Introduction}
\protect\hypertarget{introduction}{}
\begin{itemize}
\tightlist
\item
  The purpose of this vignette is a practical demonstration of reusable
  templates based upon the novel concept of data state.
\item
  This report is used as a case study for the paper: \emph{Achieving
  Analytical Fluency With Complex Data} and uses real world long term
  weather data as its source.
\item
  It is not the intention to analyse climate change, but the trends
  uncovered are striking, even in this single public domain source.
\end{itemize}
\end{frame}

\begin{frame}{Load Libraries}
\protect\hypertarget{load-libraries}{}
Load the Tidyverse libraries and other helpers before the analysis
starts.
\end{frame}

\begin{frame}[fragile]{Read The Sutton Bonnington Data}
\protect\hypertarget{read-the-sutton-bonnington-data}{}
The source data is a text file that contains the following nominal data
fields in an inconvenient format: YYYY, mm, tmax.degC, tmin.degC,
airfrost.days, rain.mm, and sun.hours.

In addition we can also deduce the following fields using our
understanding of what the data represents in the real world: place.name,
latitude, longitude, height.amsl.metre

We use the function \texttt{get.weather.data.txt()} to encapsulate all
the actions necessary to achieve this in a way that is independent of
the actual data.

\begin{Shaded}
\begin{Highlighting}[]
\NormalTok{path.to.data }\OtherTok{\textless{}{-}}
\NormalTok{  fs}\SpecialCharTok{::}\FunctionTok{path}\NormalTok{(}\StringTok{"data{-}ext"}\NormalTok{, }\StringTok{"suttonboningtondata"}\NormalTok{, }\AttributeTok{ext =} \StringTok{"txt"}\NormalTok{)}
\NormalTok{suttonboningtondata }\OtherTok{\textless{}{-}} \FunctionTok{get.weather.data.txt}\NormalTok{(path.to.data)}
\end{Highlighting}
\end{Shaded}
\end{frame}

\begin{frame}[fragile]{Introducing Datum Triples}
\protect\hypertarget{introducing-datum-triples}{}
The long datum triple format requires the explicit use of a unique
identifier for each row which we call \texttt{datumEntity}. When data is
in a wide format, as at present, the identifier is implicit as in the
row number, but this is a relative term that is affected by order. The
analysis that follows uses the advantages that are brought with an
explicit definition.
\end{frame}

\begin{frame}[fragile]{Working With Datum Triples}
\protect\hypertarget{working-with-datum-triples}{}
The datum triple is the universal starting point for all data.
Converting to this format allows us to combine data from multiple
sources, as long as the \texttt{datumEntity} only ever refers to a
unique observation. Note that this format makes no presumptions about
attributes, but it does require all data to be represented as
characters. There is no reason to keep any attribute with a value of
\texttt{NA} as the row contains no information.
\end{frame}

\begin{frame}[fragile]{Saving The Raw Data}
\protect\hypertarget{saving-the-raw-data}{}
Finally we can save the data in a machine readable format for reuse.
Although we have termed this as \texttt{data-raw}, multiple choices have
been made in its transformation into this state. It should be clear that
this \texttt{state} makes no assumptions about numbers of fields in any
observation.
\end{frame}

\begin{frame}[fragile]{Starting The Analysis With Data-Raw}
\protect\hypertarget{starting-the-analysis-with-data-raw}{}
For the purpose of this vignette we load the \texttt{data-raw} to
demonstrate the the analysis could start with multiple files using the
search style loading.
\end{frame}

\begin{frame}[fragile]{File Contents}
\protect\hypertarget{file-contents}{}
The first 13 rows of the \texttt{weather.data.triple} now look like
this:

\begin{longtable}[]{@{}
  >{\centering\arraybackslash}p{(\columnwidth - 4\tabcolsep) * \real{0.3889}}
  >{\centering\arraybackslash}p{(\columnwidth - 4\tabcolsep) * \real{0.2778}}
  >{\centering\arraybackslash}p{(\columnwidth - 4\tabcolsep) * \real{0.1806}}@{}}
\toprule
\begin{minipage}[b]{\linewidth}\centering
datumEntity
\end{minipage} & \begin{minipage}[b]{\linewidth}\centering
datumAttribute
\end{minipage} & \begin{minipage}[b]{\linewidth}\centering
datumValue
\end{minipage} \\
\midrule
\endhead
Sutton.Bonnington:1959:01 & YYYY & 1959 \\
Sutton.Bonnington:1959:01 & mm & 01 \\
Sutton.Bonnington:1959:01 & tmax.degC & 4.2 \\
Sutton.Bonnington:1959:01 & tmin.degC & -2.4 \\
Sutton.Bonnington:1959:01 & airfrost.days & 23 \\
Sutton.Bonnington:1959:01 & sun.hours & 78.8 \\
Sutton.Bonnington:1959:01 & YYYYMMDD & 1959-01-31 \\
Sutton.Bonnington:1959:01 & lattitude & 52.833 \\
Sutton.Bonnington:1959:01 & longitude & -1.25 \\
Sutton.Bonnington:1959:01 & easting & 450700 \\
Sutton.Bonnington:1959:01 & northing & 325900 \\
Sutton.Bonnington:1959:01 & height.amsl.metre & 48 \\
Sutton.Bonnington:1959:02 & YYYY & 1959 \\
\bottomrule
\end{longtable}
\end{frame}

\begin{frame}[fragile]{Prepare \texttt{data-sl}}
\protect\hypertarget{prepare-data-sl}{}
From the raw data we now prepare \texttt{data-sl} which is a loosely
defined format of convenience. All the attributes are in character
format, but may be in many different units. Re-arranging into a wider
format is helpful
\end{frame}

\begin{frame}[fragile]{Why Use Wide Format?}
\protect\hypertarget{why-use-wide-format}{}
To prepare strictly defined data to analyse the weather we select the
columns of interest and make a long format with the date as the key
column. At this point we loose any columns that are not required that
may have be present if multiple sources of \texttt{data-raw} were used.

We choose this format as it gives great flexibility with analysis and
works well with a Grammar of Graphic (GoG) approach. This contrasts with
the temptation to produce a wide format of data as one might use in a
spreadsheet analysis. The versatility of GoG will become apparent as we
proceed and see how all plots may be specified by changing the GoG
verbs.
\end{frame}

\begin{frame}{Create Data-ss}
\protect\hypertarget{create-data-ss}{}
\end{frame}

\begin{frame}{Specify Data types}
\protect\hypertarget{specify-data-types}{}
At this point we can specify data types as Date and numeric for
analysis.
\end{frame}

\begin{frame}[fragile]{Save Data-ss}
\protect\hypertarget{save-data-ss}{}
We can now save into \texttt{data-ss}.
\end{frame}

\begin{frame}[fragile]{Why Save Data-ss?}
\protect\hypertarget{why-save-data-ss}{}
Once again the data may be loaded and the analysis start from this
point. Rather than load as a named file, we demonstrate how multiple
files in \texttt{data-ss} may be loaded and combined in a simple action.
Since \texttt{data-ss} are strictly defined each file may be `stacked'
to combine into a larger data-set. If this were really the case then it
would make sense to rename the data to something more appropriate. The
named method of loading is included as comments for comparison.
\end{frame}

\begin{frame}{Load Data-ss}
\protect\hypertarget{load-data-ss}{}
\end{frame}

\begin{frame}[fragile]{A Simple Check Plot}
\protect\hypertarget{a-simple-check-plot}{}
By faceting on \texttt{datumAttribute} we can produce a separate graph
for each attribute. While is shows we have data, each graph has
different units, so the scales do not make sense.
\end{frame}

\begin{frame}[fragile]{Simple Check Plot}
\protect\hypertarget{simple-check-plot}{}
By faceting on \texttt{datumAttribute} we can produce a separate graph
for each attribute. While is shows we have data, each graph has
different units, so the scales do not make sense.

\begin{center}\includegraphics[width=0.7\linewidth]{Sutton_Bonnington_weather_files/figure-beamer/SimpleCheckPlot-1} \end{center}
\end{frame}

\begin{frame}{Normalising With Z-scores}
\protect\hypertarget{normalising-with-z-scores}{}
However, if we use z-scores instead then each plot is normalised against
zero and the standard deviation.
\end{frame}

\begin{frame}{Plotting With Z-scores}
\protect\hypertarget{plotting-with-z-scores}{}
Rather than 5 separate plots, we can use a single graph to plot a smooth
loess regression for each attribute.

\begin{center}\includegraphics[width=0.7\linewidth]{Sutton_Bonnington_weather_files/figure-beamer/NormalisingWithZScore-1} \end{center}
\end{frame}

\begin{frame}{Checking Data Distribution}
\protect\hypertarget{checking-data-distribution}{}
\end{frame}

\begin{frame}{Q-Q Plot Z-score Compared To Normal Distribution}
\protect\hypertarget{q-q-plot-z-score-compared-to-normal-distribution}{}
This looks good but it is predicated upon the deviations being normally
distributed about the mean.

\begin{center}\includegraphics[width=0.7\linewidth]{Sutton_Bonnington_weather_files/figure-beamer/CheckingDataDistributionQQ-1} \end{center}
\end{frame}

\begin{frame}{Assumptions}
\protect\hypertarget{assumptions}{}
Our assumption or a normal distribution is reasonably valid, but not
perfect. As we are looking at weather data, it might be nice to look at:
Annual Total Rainfall, Annual Air Frost days, Annual Maximum
Temperature, and Annual Minimum Temperature. Again GoG comes to the
rescue and we can quickly produce the following plots.
\end{frame}

\begin{frame}{Weather Plots In Native Units}
\protect\hypertarget{weather-plots-in-native-units}{}
For convenience, the plot routine has been written as a function so we
can get all the plots quickly by reusing the code and just changing the
selection attribute.
\end{frame}

\begin{frame}{Filter The Data}
\protect\hypertarget{filter-the-data}{}
\end{frame}

\begin{frame}{Plot Total Airfrost Days}
\protect\hypertarget{plot-total-airfrost-days}{}
\begin{center}\includegraphics[width=0.7\linewidth]{Sutton_Bonnington_weather_files/figure-beamer/PlotAirFrost-1} \end{center}
\end{frame}

\begin{frame}{Plot Maximum Temperature}
\protect\hypertarget{plot-maximum-temperature}{}
\begin{center}\includegraphics[width=0.7\linewidth]{Sutton_Bonnington_weather_files/figure-beamer/PlotMaxTemp-1} \end{center}
\end{frame}

\begin{frame}{Plot Minimum Temperature}
\protect\hypertarget{plot-minimum-temperature}{}
\begin{center}\includegraphics[width=0.7\linewidth]{Sutton_Bonnington_weather_files/figure-beamer/PlotMinTemp-1} \end{center}
\end{frame}

\begin{frame}{Plot Total Rainfail}
\protect\hypertarget{plot-total-rainfail}{}
\begin{center}\includegraphics[width=0.7\linewidth]{Sutton_Bonnington_weather_files/figure-beamer/PlotTotalRainfall-1} \end{center}
\end{frame}

\begin{frame}{Conclusion}
\protect\hypertarget{conclusion}{}
This vignette demonstrates the versatility of using data state in
conjunction with GoG.
\end{frame}

\end{document}
