% !TeX program = pdfLaTeX
\documentclass[smallextended]{svjour3}       % onecolumn (second format)
%\documentclass[twocolumn]{svjour3}          % twocolumn
%
\smartqed  % flush right qed marks, e.g. at end of proof
%
\usepackage{amsmath}
\usepackage{graphicx}
\usepackage[utf8]{inputenc}

\usepackage[hyphens]{url} % not crucial - just used below for the URL
\usepackage{hyperref}

%
% \usepackage{mathptmx}      % use Times fonts if available on your TeX system
%
% insert here the call for the packages your document requires
%\usepackage{latexsym}
% etc.
%
% please place your own definitions here and don't use \def but
% \newcommand{}{}
%
% Insert the name of "your journal" with
% \journalname{myjournal}
%

%% load any required packages here



% tightlist command for lists without linebreak
\providecommand{\tightlist}{%
  \setlength{\itemsep}{0pt}\setlength{\parskip}{0pt}}



\begin{document}


\title{Achieving Analytical Fluency With Complex Data \thanks{Grants or
other notes about the article that should go on the front page should be
placed here. General acknowledgments should be placed at the end of the
article.} }
 \subtitle{Do you have a subtitle? If so, write it here} 

    \titlerunning{Analytical Fluency}

\author{  Paul J Palmer \and  Michael Henshaw \and  Russell Lock \and  }

    \authorrunning{ P.J. Palmer, M. Henshaw, R.L. Lock }

\institute{
        Paul J Palmer \at
     Department of YYY, University of XXX \\
     \email{\href{mailto:abc@def}{\nolinkurl{abc@def}}}  %  \\
%             \emph{Present address:} of F. Author  %  if needed
    \and
        Michael Henshaw \at
     Department of ZZZ, University of WWW \\
     \email{\href{mailto:djf@wef}{\nolinkurl{djf@wef}}}  %  \\
%             \emph{Present address:} of F. Author  %  if needed
    \and
        Russell Lock \at
     Department of ZZZ, University of WWW \\
     \email{\href{mailto:djf@wef}{\nolinkurl{djf@wef}}}  %  \\
%             \emph{Present address:} of F. Author  %  if needed
    \and
    }

\date{Received: date / Accepted: date}
% The correct dates will be entered by the editor


\maketitle

\begin{abstract}
Scientific analysis is formally presented as a rigid process typically
comprising: Review; Theory; Research question; Methodology; Experiment;
Analysis, Evaluation; and Conclusions.

We do, however, question whether established methods for managing and
analysing data are still appropriate for ``Big Data'', by which we mean
data that is too big to be conveniently manipulated by manually
intensive methods.

This presents two questions which we seek to address here: the first is
to justify why there is a need for new analytical techniques, given that
the existing ones still work; the second is to show how an abstract
perception of data impacts the analytical process.

An example of the motivation for change is illustrated by the ``Grammar
of Graphics'' (GoG) paradigm. GoG uses combinations of transforms to
generate every possible graphic and tabulation from data presented in a
suitable state allowing for a radical change in the analytic workflow,
while still preserving the goals of scientific analysis.

By framing this problem as one of transforming \emph{data state}, we can
mathematically describe general properties allowing the creation of
re-usable code templates for data preparation. Coupled with ``literate
programming'' techniques we show that this approach enables analytically
fluent analysis of complex data.
\\
\keywords{
        key \and
        dictionary \and
        word \and
    }


\end{abstract}


\def\spacingset#1{\renewcommand{\baselinestretch}%
{#1}\small\normalsize} \spacingset{1}


\hypertarget{intro}{%
\section{Introduction}\label{intro}}

Your text comes here. Separate text sections with \cite{Mislevy06Cog}.

\hypertarget{sec:1}{%
\section{Section title}\label{sec:1}}

Text with citations by \cite{Galyardt14mmm}.

\hypertarget{sec:2}{%
\subsection{Subsection title}\label{sec:2}}

as required. Don't forget to give each section and subsection a unique
label (see Sect. \ref{sec:1}).

\hypertarget{paragraph-headings}{%
\paragraph{Paragraph headings}\label{paragraph-headings}}

Use paragraph headings as needed.

\begin{align}
a^2+b^2=c^2
\end{align}


\bibliographystyle{spphys}
\bibliography{bibliography.bib}


\end{document}
