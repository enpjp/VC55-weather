% !TeX program = pdfLaTeX
\documentclass[smallextended]{svjour3}       % onecolumn (second format)
%\documentclass[twocolumn]{svjour3}          % twocolumn
%
\smartqed  % flush right qed marks, e.g. at end of proof
%
\usepackage{amsmath}
\usepackage{graphicx}
\usepackage[utf8]{inputenc}

\usepackage[hyphens]{url} % not crucial - just used below for the URL
\usepackage{hyperref}

%
% \usepackage{mathptmx}      % use Times fonts if available on your TeX system
%
% insert here the call for the packages your document requires
%\usepackage{latexsym}
% etc.
%
% please place your own definitions here and don't use \def but
% \newcommand{}{}
%
% Insert the name of "your journal" with
% \journalname{myjournal}
%

%% load any required packages here



% tightlist command for lists without linebreak
\providecommand{\tightlist}{%
  \setlength{\itemsep}{0pt}\setlength{\parskip}{0pt}}

% From pandoc table feature
\usepackage{longtable,booktabs,array}
\usepackage{calc} % for calculating minipage widths
% Correct order of tables after \paragraph or \subparagraph
\usepackage{etoolbox}
\makeatletter
\patchcmd\longtable{\par}{\if@noskipsec\mbox{}\fi\par}{}{}
\makeatother
% Allow footnotes in longtable head/foot
\IfFileExists{footnotehyper.sty}{\usepackage{footnotehyper}}{\usepackage{footnote}}
\makesavenoteenv{longtable}

% Pandoc citation processing
\newlength{\cslhangindent}
\setlength{\cslhangindent}{1.5em}
\newlength{\csllabelwidth}
\setlength{\csllabelwidth}{3em}
\newlength{\cslentryspacingunit} % times entry-spacing
\setlength{\cslentryspacingunit}{\parskip}
% for Pandoc 2.8 to 2.10.1
\newenvironment{cslreferences}%
  {}%
  {\par}
% For Pandoc 2.11+
\newenvironment{CSLReferences}[2] % #1 hanging-ident, #2 entry spacing
 {% don't indent paragraphs
  \setlength{\parindent}{0pt}
  % turn on hanging indent if param 1 is 1
  \ifodd #1
  \let\oldpar\par
  \def\par{\hangindent=\cslhangindent\oldpar}
  \fi
  % set entry spacing
  \setlength{\parskip}{#2\cslentryspacingunit}
 }%
 {}
\usepackage{calc}
\newcommand{\CSLBlock}[1]{#1\hfill\break}
\newcommand{\CSLLeftMargin}[1]{\parbox[t]{\csllabelwidth}{#1}}
\newcommand{\CSLRightInline}[1]{\parbox[t]{\linewidth - \csllabelwidth}{#1}\break}
\newcommand{\CSLIndent}[1]{\hspace{\cslhangindent}#1}

% Jese tex
\begin{document}


\title{Achieving Analytical Fluency With Complex Data \thanks{The authors acknowledge EPSRC grant: EP/R513088/1 for funding the doctoral research on which this work is based.} }
 \subtitle{Do you have a subtitle? If so, write it here} 

    \titlerunning{Analytical Fluency}

\author{  Paul J Palmer \and  Michael Henshaw \and  Russell Lock \and  }

    \authorrunning{ P.J. Palmer, M. Henshaw, R.L. Lock }

\institute{
        Paul J Palmer \at
     Department of YYY, University of XXX \\
     \email{\href{mailto:abc@def}{\nolinkurl{abc@def}}}  %  \\
%             \emph{Present address:} of F. Author  %  if needed
    \and
        Michael Henshaw \at
     Department of ZZZ, University of WWW \\
     \email{\href{mailto:djf@wef}{\nolinkurl{djf@wef}}}  %  \\
%             \emph{Present address:} of F. Author  %  if needed
    \and
        Russell Lock \at
     Department of ZZZ, University of WWW \\
     \email{\href{mailto:djf@wef}{\nolinkurl{djf@wef}}}  %  \\
%             \emph{Present address:} of F. Author  %  if needed
    \and
    }

\date{Received: date / Accepted: date}
% The correct dates will be entered by the editor


\maketitle

\begin{abstract}
Scientific analysis is formally presented as a rigid process typically
Experiment; Analysis, Evaluation; and Conclusions.
comprising: Review; Theory; Research question; Methodology;
We do, however, question whether established methods for managing and
analysing data are still appropriate for ``Big Data'', by which we mean
data that is too big to be conveniently manipulated by manually
intensive methods.
is to justify why there is a need for new analytical techniques, given
that the existing ones still work; the second is to show how an
abstract perception of data impacts the analytical process.
This presents two questions which we seek to address here: the first
An example of the motivation for change is illustrated by the ``Grammar
of Graphics'' (GoG) paradigm. GoG uses combinations of transforms to
generate every possible graphic and tabulation from data presented in
a suitable state allowing for a radical change in the analytic
workflow, while still preserving the goals of scientific analysis.
By framing this problem as one of transforming \emph{data state}, we can
mathematically describe general properties allowing the creation of
re-usable code templates for data preparation. Coupled with ``literate
programming'' techniques we show that this approach enables
analytically fluent analysis of complex data.
\\
\keywords{
        key \and
        dictionary \and
        word \and
    }


\end{abstract}


\def\spacingset#1{\renewcommand{\baselinestretch}%
{#1}\small\normalsize} \spacingset{1}


\hypertarget{sec:introduction}{%
\section{Introduction}\label{sec:introduction}}

Scientific analysis is formally presented as a rigid process typically
comprising: Review; Theory; Research question; Methodology; Experiment;
Analysis, Evaluation; and Conclusions. This is a powerful framework that
has stood the test of time which is the underlying format of countless
academic publications and there is no reason to think that this format
will change.

If we look deeper, we can start to separate the actual process of
scientific discovery from the way in which it is reported. The
documentation of research and analysis that has already been completed,
is very different to an exploration where the outcomes are as yet
unknown. Nobel Prize winner Szent-Györgyi (1972) discussed this issue in
1972, long before the age of computers and ``Big Data''{[}\^{}2{]} introduced
computational and data intensive research methods~(Szent-Györgyi 1972).
He described the Apollonian as following a formally prescribed research
path while a Dionysian explores the unknown. However, research and
reporting are necessarily linked and one should be a clear and true
interpretation of the other.

We argue here that this presumption the analytical process should follow
the final method of presentation is a hindrance to the adoption of new
analytical techniques. The term ``reproducible research'' has been coined
to describe this linkage, without defining how it may be achieved
leading to discussion on the topic by~Stodden, Leisch, and Peng (2014; Lewis, Vander Wal, and Fifield 2018; Leeper 2014; Gentleman and Temple Lang 2007; Peng 2015; Drummond 2018) and
others.

Two questions arise from these discussions which we seek to address in
the following sections: the first is to justify why there is a need for
new analytical techniques, given that the existing ones still work? The
second is: how can changing our abstract perception of data beneficially
impact the analytical process?

Both the need and benefit are illustrated by the ``Grammar of Graphics''
(GoG) paradigm (Wilkinson 2010). GoG uses combinations of transforms to
generate every possible graphic and tabulation from data that is
presented in a suitable state. Thus, a researcher planning to use GoG
will focus on saving research data in suitable state, so she can make
many dynamic visualisations as work progresses, rather than waiting for
the end of the data collation phase.

This contrasts with the typical approach where difficulties with data
intensive analysis are often retrospectively blamed on imperfections
within the data that are seen as requiring ``cleaning'' before analysis
can begin. It is unfortunate that the popular spreadsheet encourages a
manually intensive formatting of data and the use of visualisation tools
that do not scale well with data size
(Mack et al. 2018; Bishop 2013; Powell, Baker, and Lawson 2008).

Our alternative approach is consider data as observations of the real
world that are to be interpreted through analysis, rather than formatted
tables meeting arbitrary technical specifications.{[}\^{}3{]} This also expands
on an idea hinted at in the GoG example above: there is no need to
rigidly define data format in advance since the real world is clearly
independent of the methods we choose to observe and record. If we
believe that the knowledge we seek is encapsulated within a set of
observations, then we can choose a viewpoint where we see data as
starting in an initial \emph{inconvenient state}. It follows that we must
transform it into a \emph{convenient state} for analysis. By constructing a
working theory supported with a mathematical description, we show that
the transformation of \emph{data state} can be generalised to allow the
creation of re-usable code templates. To explain how re-framing the
problem of \emph{imperfect data} in terms of \emph{data state} can make such a
difference to the way in which we approach the analytical process, we
must first construct a methodology that takes into account the nature of
data.

\hypertarget{refs}{}
\begin{CSLReferences}{1}{0}
\leavevmode\vadjust pre{\hypertarget{ref-Bishop2013}{}}%
Bishop, Katrina. 2013. {``{Spreadsheet blunders costing business billions}.''} Englewood Cliffs, New Jersey,USA. \url{https://www.cnbc.com/id/100923538}.

\leavevmode\vadjust pre{\hypertarget{ref-Drummond2018}{}}%
Drummond, Chris. 2018. {``{Reproducible research: a minority opinion}.''} \emph{Journal of Experimental and Theoretical Artificial Intelligence} 30 (1): 1--11. \url{https://doi.org/10.1080/0952813X.2017.1413140}.

\leavevmode\vadjust pre{\hypertarget{ref-Gentleman2007}{}}%
Gentleman, Robert, and Duncan Temple Lang. 2007. {``{Statistical Analyses and Reproducible Research}.''} \emph{Journal of Computational and Graphical Statistics} 16 (1): 1--23. \url{https://doi.org/10.1198/106186007X178663}.

\leavevmode\vadjust pre{\hypertarget{ref-Leeper2014}{}}%
Leeper, J., Thomas. 2014. {``{Archiving Reproducible Research with R and Dataverse}.''} \emph{The R Journal} 6 (1): 151. \url{https://doi.org/10.32614/RJ-2014-015}.

\leavevmode\vadjust pre{\hypertarget{ref-Lewis2018a}{}}%
Lewis, Keith P., Eric Vander Wal, and David A. Fifield. 2018. {``{Wildlife biology, big data, and reproducible research}.''} \emph{Wildlife Society Bulletin} 42 (1): 172--79. \url{https://doi.org/10.1002/wsb.847}.

\leavevmode\vadjust pre{\hypertarget{ref-Mack2018}{}}%
Mack, Kelly, John Lee, Kevin Chang, Karrie Karahalios, and Aditya Parameswaran. 2018. {``{Characterizing Scalability Issues in Spreadsheet Software using Online Forums}.''} University of Illinois. \href{http://arxiv.org/abs/1801.03829\%20https://arxiv.org/pdf/1801.03829.pdf}{http://arxiv.org/abs/1801.03829 https://arxiv.org/pdf/1801.03829.pdf}.

\leavevmode\vadjust pre{\hypertarget{ref-Peng2015}{}}%
Peng, Roger. 2015. {``{The reproducibility crisis in science: A statistical counterattack}.''} \emph{Significance} 12 (3): 30--32. \url{https://doi.org/10.1111/j.1740-9713.2015.00827.x}.

\leavevmode\vadjust pre{\hypertarget{ref-Powell2008}{}}%
Powell, Stephen G., Kenneth R. Baker, and Barry Lawson. 2008. {``{A critical review of the literature on spreadsheet errors}.''} \emph{Decision Support Systems} 46 (1): 128--38. \url{https://doi.org/10.1016/j.dss.2008.06.001}.

\leavevmode\vadjust pre{\hypertarget{ref-Stodden2014}{}}%
Stodden, Victoria, Friedrich Leisch, and Roger D. Peng. 2014. \emph{{Implementing reproducible research}}. CRC Press/Taylor; Francis.

\leavevmode\vadjust pre{\hypertarget{ref-Szent-Gyorgyi1972}{}}%
Szent-Györgyi, Albert. 1972. {``{Dionysians and apollonians}.''} \emph{Science} 176 (4038): 966. \url{https://doi.org/10.1126/science.176.4038.966}.

\leavevmode\vadjust pre{\hypertarget{ref-Wilkinson2010}{}}%
Wilkinson, Leland. 2010. \emph{{The grammar of graphics}}. \url{https://doi.org/10.1002/wics.118}.

\end{CSLReferences}


\bibliographystyle{spphys}
\bibliography{library.bib}


\end{document}
